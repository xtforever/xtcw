\input wbuildmac.tex
\Class{Wsel2}\Publicvars
\Table{Wsel2}
XtNvalue&XtCValue&int &0 \cr
XtNlocked&XtCLocked&Bool &0 \cr
XtNnames&XtCNames&StringMArray &0 \cr
XtNcornerRoundPercent&XtCCornerRoundPercent&int &0 \cr
XtNinnerBorder&XtCInnerBorder&int &0 \cr
XtNupdate&XtCUpdate&int &0 \cr
XtNhgap&XtCHgap&Distance &0 \cr
\endTable
\Section
\Publicvar{XtNvalue}
int  value = 0 

\Section
\Publicvar{XtNlocked}
Bool  locked = 0 

\Section
\Publicvar{XtNnames}
StringMArray  names = 0 

\Section
\Publicvar{XtNcornerRoundPercent}
int  cornerRoundPercent = 0 

\Section
\Publicvar{XtNinnerBorder}
int  innerBorder = 0 

\Section
\Publicvar{XtNupdate}
int  update = 0 

\Section
\Publicvar{XtNhgap}
Distance  hgap = 0 

\End\Table{Wheel}
XtNxftFont&XtCXFtFont&XftFont&"Sans-22"\cr
XtNcallback&XtCCallback&Callback&NULL \cr
XtNbg{\underline}norm&XtCBg{\underline}norm&Pixel&"lightblue"\cr
XtNbg{\underline}sel&XtCBg{\underline}sel&Pixel&"yellow"\cr
XtNbg{\underline}hi&XtCBg{\underline}hi&Pixel&"red"\cr
XtNfg{\underline}norm&XtCFg{\underline}norm&Pixel&"black"\cr
XtNfg{\underline}sel&XtCFg{\underline}sel&Pixel&"green"\cr
XtNfg{\underline}hi&XtCFg{\underline}hi&Pixel&"white"\cr
XtNuser{\underline}data&XtCUser{\underline}data&Int &0 \cr
XtNfocus{\underline}group&XtCFocus{\underline}group&String &""\cr
XtNstate&XtCState&Int &0 \cr
XtNregister{\underline}focus{\underline}group&XtCRegister{\underline}focus{\underline}group&Boolean &True \cr
\endTable
\Table{Core}
XtNx&XtCX&Position &0 \cr
XtNy&XtCY&Position &0 \cr
XtNwidth&XtCWidth&Dimension &0 \cr
XtNheight&XtCHeight&Dimension &0 \cr
borderWidth&XtCBorderWidth&Dimension &0 \cr
XtNcolormap&XtCColormap&Colormap &NULL \cr
XtNdepth&XtCDepth&Int &0 \cr
destroyCallback&XtCDestroyCallback&XTCallbackList &NULL \cr
XtNsensitive&XtCSensitive&Boolean &True \cr
XtNtm&XtCTm&XTTMRec &NULL \cr
ancestorSensitive&XtCAncestorSensitive&Boolean &False \cr
accelerators&XtCAccelerators&XTTranslations &NULL \cr
borderColor&XtCBorderColor&Pixel &0 \cr
borderPixmap&XtCBorderPixmap&Pixmap &NULL \cr
background&XtCBackground&Pixel &0 \cr
backgroundPixmap&XtCBackgroundPixmap&Pixmap &NULL \cr
mappedWhenManaged&XtCMappedWhenManaged&Boolean &True \cr
XtNscreen&XtCScreen&Screen *&NULL \cr
\endTable
\Imports
\Section
\Code
{\incl} {\langle}X11/Xregion.h{\rangle}\endCode


\Section
\Code
{\incl} {\langle}X11/Xft/Xft.h{\rangle}\endCode


\Section
\Code
{\incl} "converters.h"\endCode


\Section
\Code
{\incl} {\langle}X11/Xmu/Converters.h{\rangle}\endCode


\Section
\Code
{\incl} "mls.h"\endCode


\End\Privatevars
\Section
\Code
int  last{\underline}value\endCode


\Section
\Code
int  total{\underline}width\endCode


\Section
\Code
int  max{\underline}value\endCode


\Section
\Code
int  pm{\underline}w\endCode


\Section
\Code
int  pm{\underline}h\endCode


\Section
\Code
Pixmap  pixmap\endCode


\Section
\Code
XftDraw * draw\endCode


\Section
\Code
GC  gc{\underline}copy\endCode


\Section
\Code
XRectangle  r{\underline}label\endCode


\Section
\Code
Bool  do{\underline}init\endCode


\Section
\Code
int  rects\endCode


\Section
\Code
 debug\endCode


\End\Classvars
\Section
\Code
compress{\underline}exposure = XtExposeCompressSeries \endCode


\Section
\Code
visible{\underline}interest = True \endCode


\End\Methods
\Section
\Code
initialize(Widget  request, {\dollar}, ArgList  args, Cardinal * num{\underline}args)
{\lbrace}
        {\dollar}do{\underline}init = 1;
        {\dollar}pixmap = 0;
        {\dollar}draw = 0;
        {\dollar}gc{\underline}copy =  XtGetGC({\dollar}, 0,0 );
        {\dollar}rects = m{\underline}create(2,sizeof(XRectangle));

        if( {\dollar}names == 0 ) {\dollar}max{\underline}value=0;
        else {\dollar}max{\underline}value = m{\underline}len({\dollar}names);

        int w, h;
        prefered{\underline}size( {\dollar}, {\ampersand}w, {\ampersand}h );
        if( {\dollar}width == 0 ) {\dollar}width = w;
        if( {\dollar}height == 0 ) {\dollar}height = h;
{\rbrace}\endCode


\Section
falls noch keine initialisierung stattgefunden hat,
  alles initialisieren und neu zeichnen.
  sonst nur den angegebenen ausschnitt neu zeichnen,
  falls uns der XServer keinen bereich angiebt
  wird das ganze fenster neu gezeichnet



\Code
expose({\dollar}, XEvent * event, Region  region)
{\lbrace}
  TRACE(debug,"{\percent}s", {\dollar}name );

  if( {\dollar}do{\underline}init ) {\lbrace}
      alloc{\underline}pixmap({\dollar});
      clear{\underline}pixmap({\dollar});
      draw{\underline}text({\dollar});
      copy{\underline}pixmap{\underline}to{\underline}window({\dollar});
      {\dollar}do{\underline}init = 0;
      return;
  {\rbrace}

  copy{\underline}region{\underline}to{\underline}window({\dollar},region);
{\rbrace}\endCode


\Section
\Code
realize({\dollar}, XtValueMask * mask, XSetWindowAttributes * attributes)
{\lbrace}
     TRACE(debug, "{\percent}s: {\percent}dx{\percent}d", {\dollar}name, {\dollar}width, {\dollar}height );
     XtCreateWindow({\dollar}, (unsigned int) InputOutput,
                          (Visual *) CopyFromParent, *mask, attributes);
     reshape{\underline}widget({\dollar});
{\rbrace}\endCode


\Section
marker setzen damit das folgende expose das Window
  neu zeichnen kann, resize wird beim initialisieren der application
  vor realize aufgerufen daher die abfrage ob es sicher ist das
  window zu bearbeiten.


\Code
resize({\dollar})
{\lbrace}
    if( XtIsRealized({\dollar}) ) reshape{\underline}widget({\dollar});
    {\dollar}do{\underline}init=1;
{\rbrace}\endCode


\Section
\Code
destroy({\dollar})
{\lbrace}
  if( {\dollar}draw) XftDrawDestroy( {\dollar}draw );
  XtReleaseGC({\dollar},{\dollar}gc{\underline}copy);
  free{\underline}pixmap({\dollar});
{\rbrace}\endCode


\Section
ARGS: old, request, $, args, num_args_ptr


\Code
Boolean  set{\underline}values(Widget  old, Widget  request, {\dollar}, ArgList  args, Cardinal * num{\underline}args)
{\lbrace}
    /* window erst neu zeichnen wenn es sichtbar geworden ist */
    if( !{\dollar}visible ) {\lbrace} {\dollar}do{\underline}init = 1; return 0; {\rbrace}

  /* falls das widget noch nicht initialisiert ist
       oder ein resize stattgefunden hat wird bei
       expose alles neu gezeichnet. daher keine
       aktion notwendig
  */
  if( {\dollar}do{\underline}init ) return 1; /* call expose */

  if( {\dollar}border{\underline}width != {\dollar}old{\dollar}border{\underline}width )
  {\lbrace}
        reshape{\underline}widget({\dollar});
  {\rbrace}


  /* bei einem state change wird alles neu gezeichnet,
     also gibt es keinen grund noch die anderen änderungen
     zu überprüfen
  */
  if( ({\dollar}state != {\dollar}old{\dollar}state) {\bar}{\bar}  ({\dollar}old{\dollar}update != {\dollar}update)  ) {\lbrace}
      clear{\underline}pixmap({\dollar});
      draw{\underline}text({\dollar});
      copy{\underline}pixmap{\underline}to{\underline}window({\dollar});
      return 0;
  {\rbrace}

  /* das aktive label hat sich geändert, dann muss das alte
     label gelöscht werden und das neue gezeichnet.
     da keine weiteren attribute vorgesehen sind
     kann hier beendet werden
  */
  if( ({\dollar}value != {\dollar}old{\dollar}value) ) {\lbrace}
      clear{\underline}old{\underline}text({\dollar});
      draw{\underline}text({\dollar});
      copy{\underline}pixmap{\underline}to{\underline}window({\dollar});
      return 0;
  {\rbrace}

  return 0;
{\rbrace}\endCode


\Section
parameter list ($, int cmd, int val)


\Code
int  exec{\underline}command({\dollar}, int  cmd, int  val)
{\lbrace}
  return 0;
{\rbrace}\endCode


\Section
\Code
Bool  cache{\underline}hit({\dollar})
{\lbrace}
    return 0;
{\rbrace}\endCode


\Section
\Code
void  minimal{\underline}size({\dollar}, int * w0, int * h0)
{\lbrace}
        XGlyphInfo extents;
        Display *dpy = XtDisplay({\dollar});
        int i,w = 0, h;
        h = {\dollar}xftFont-{\rangle}ascent + {\dollar}xftFont-{\rangle}descent;
        char *s;

        int len = m{\underline}len({\dollar}names);
        m{\underline}clear( {\dollar}rects );
        XRectangle *r;

        for(i=0; i {\langle} len;i++) {\lbrace}
                 s = get{\underline}name({\dollar},i);
                 XftTextExtentsUtf8(dpy, {\dollar}xftFont, (FcChar8*)  s,
                       strlen(s), {\ampersand}extents);
                 w += extents.xOff;
                 r = m{\underline}add({\dollar}rects);
                 r-{\rangle}width = extents.xOff;
                 r-{\rangle}height = h;
        {\rbrace}
        {\dollar}total{\underline}width = w;
        if( len ) w += (len-1) * {\dollar}hgap;
        if( w0 ) *w0 = w;
        if( h0 ) *h0 = h;
{\rbrace}\endCode


\Section
\Code
void  prefered{\underline}size({\dollar}, int * w0, int * h0)
{\lbrace}
    minimal{\underline}size({\dollar},w0,h0);
    if( w0 ) *w0 +=2 * {\dollar}innerBorder;
    if( h0 ) *h0 +=2 * {\dollar}innerBorder;
{\rbrace}\endCode


\End\Utilities
\Section
\Code
clear{\underline}pixmap({\dollar})
{\lbrace}
        TRACE(debug, "{\percent}s", {\dollar}name );
  XFillRectangle(XtDisplay({\dollar}),{\dollar}pixmap, {\dollar}gc[{\dollar}state], 0,0,{\dollar}width,{\dollar}height);
{\rbrace}\endCode


\Section
\Code
copy{\underline}pixmap{\underline}to{\underline}window({\dollar})
{\lbrace}
    if( {\dollar}visible )
       XCopyArea(XtDisplay({\dollar}),{\dollar}pixmap, XtWindow({\dollar}), {\dollar}gc{\underline}copy,
            0,0, {\dollar}width,{\dollar}height,  /* source pixmap */
            0,0  ); /* target window x,y */

{\rbrace}\endCode


\Section
\Code
copy{\underline}region{\underline}to{\underline}window({\dollar}, Region  region)
{\lbrace}
    TRACE(debug,"{\percent}s", {\dollar}name );
    int x=0,y=0,w,h;
    if( region ) {\lbrace}
          Box *b = {\ampersand} region-{\rangle}extents;

          w = b-{\rangle}x2 - b-{\rangle}x1 +1;
          h = b-{\rangle}y2 - b-{\rangle}y1 +1;
          x = b-{\rangle}x1;
          y = b-{\rangle}y1;
  {\rbrace} else {\lbrace}
          w = {\dollar}width;
          h = {\dollar}height;
  {\rbrace}

  TRACE(debug, "{\percent}s: copy extents {\percent}dx{\percent}d,{\percent}dx{\percent}d",
        {\dollar}name, x,y,w,h );

  XCopyArea(XtDisplay({\dollar}),{\dollar}pixmap, XtWindow({\dollar}), {\dollar}gc{\underline}copy,
            x,y, w,h,  /* source pixmap */
            x,y  ); /* target window x,y */
{\rbrace}\endCode


\Section
\Code
draw{\underline}text({\dollar})
{\lbrace}
    TRACE(debug,"{\percent}s", {\dollar}name );
    if( {\dollar}max{\underline}value == 0 ) return;

    int w,h;
    minimal{\underline}size( {\dollar}, {\ampersand}w, {\ampersand}h );
    if( w {\rangle} {\dollar}width ) w={\dollar}width;
    if( h {\rangle} {\dollar}height ) h={\dollar}height;

    int x = ({\dollar}width - w) / 2;
    int y = ({\dollar}height -h) / 2 ;
    char *s;
    XRectangle *r;
    XftColor *fg,*bg;

    int p;
    m{\underline}foreach( {\dollar}rects, p, r ) {\lbrace}

        if( p == {\dollar}value ) {\lbrace}
            fg = {\dollar}xft{\underline}col + COLOR{\underline}FG{\underline}SEL;
            bg = {\dollar}xft{\underline}col + COLOR{\underline}BG{\underline}SEL;
        {\rbrace}
        else {\lbrace}
            fg = {\dollar}xft{\underline}col + COLOR{\underline}FG{\underline}NORM;
            bg = {\dollar}xft{\underline}col + COLOR{\underline}BG{\underline}NORM;
        {\rbrace}

        s = get{\underline}name({\dollar},p);
        r-{\rangle}x = x; r-{\rangle}y = y;

        XftDrawRect({\dollar}draw,bg, r-{\rangle}x, r-{\rangle}y, r-{\rangle}width, r-{\rangle}height );
        XftDrawStringUtf8({\dollar}draw, fg, {\dollar}xftFont,
                             x,y+ {\dollar}xftFont-{\rangle}ascent, (FcChar8*)s, strlen(s) );

        x+= {\dollar}hgap + r-{\rangle}width;
    {\rbrace}

    {\dollar}last{\underline}value = {\dollar}value;

{\rbrace}\endCode


\Section
\Code
clear{\underline}old{\underline}text({\dollar})
{\lbrace}
    clear{\underline}pixmap({\dollar});
{\rbrace}\endCode


\Section
\Code
reshape{\underline}widget({\dollar})
{\lbrace}
        int w;
        if( {\dollar}cornerRoundPercent {\rangle}0 {\ampersand}{\ampersand} {\dollar}cornerRoundPercent {\langle} 100 ) {\lbrace}
            w = Min({\dollar}height,{\dollar}width);
            w = w * {\dollar}cornerRoundPercent / 100;
            XmuReshapeWidget( {\dollar}, XmuShapeRoundedRectangle, w, w );
        {\rbrace}
{\rbrace}\endCode


\Section
create a pixmap and update or create
  Xft $draw Context.


\Code
alloc{\underline}pixmap({\dollar})
{\lbrace}
    Display *dpy = XtDisplay({\dollar});

    /* falls eine pixmap mit ausreichender größe vorhanden ist,
       keine neue pixmap erzeugen, sondern alte weiterverwenden */
    if( {\dollar}pixmap {\ampersand}{\ampersand}  ({\dollar}pm{\underline}w {\rangle}= {\dollar}width) {\ampersand}{\ampersand} ({\dollar}pm{\underline}h {\rangle}= {\dollar}height) )
        return;

    TRACE(debug,"{\percent}s", {\dollar}name );

    if( {\dollar}pixmap ) XFreePixmap(dpy,{\dollar}pixmap);

    {\dollar}pixmap = XCreatePixmap(dpy, XtWindow({\dollar}), {\dollar}width, {\dollar}height,
                            DefaultDepth(dpy, DefaultScreen(dpy)));
    {\dollar}pm{\underline}w = {\dollar}width; {\dollar}pm{\underline}h = {\dollar}height;

    if( {\dollar}draw ) {\lbrace}
        XftDrawChange({\dollar}draw, {\dollar}pixmap );
    {\rbrace}
    else {\lbrace}
        {\dollar}draw = XftDrawCreate(dpy, {\dollar}pixmap,
                       DefaultVisual(dpy, DefaultScreen(dpy)), None);
    {\rbrace}
{\rbrace}\endCode


\Section
\Code
free{\underline}pixmap({\dollar})
{\lbrace}
        if( !{\dollar}pixmap ) return;
        Display *dpy = XtDisplay({\dollar});
        XFreePixmap(dpy,{\dollar}pixmap);
        {\dollar}pixmap = 0;
    {\rbrace}\endCode


\Section
\Code
String  get{\underline}name({\dollar}, int  p)
{\lbrace}
  if( {\dollar}max{\underline}value == 0 ) return "";
  if( p {\langle} 0 ) p = {\dollar}max{\underline}value -1;
  else if( p {\rangle}= {\dollar}max{\underline}value ) p = 0;
  return STR({\dollar}names,p);
{\rbrace}\endCode


\End\bye
