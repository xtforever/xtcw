\input wbuildmac.tex
\Class{TermEd}\Publicvars
\Table{TermEd}
XtNgWidth&XtCGWidth&int &40 \cr
XtNgHeight&XtCGHeight&int &25 \cr
XtNsize&XtCSize&int &8 \cr
XtNselTime&XtCSelTime&int &400 \cr
XtNtext&XtCText&String &"Hello World{\backslash}nHow are you?"\cr
XtNfg&XtCFg&XftColor&"White"\cr
XtNbg&XtCBg&XftColor&"Darkgreen"\cr
XtNcbg&XtCCbg&XftColor&"Red"\cr
XtNcfg&XtCCfg&XftColor&"Green"\cr
XtNhost&XtCHost&String &"localhost"\cr
XtNport&XtCPort&String &"10002"\cr
\endTable
\Section
\Publicvar{XtNgWidth}
int  gWidth = 40 

\Section
\Publicvar{XtNgHeight}
int  gHeight = 25 

\Section
\Publicvar{XtNsize}
int  size = 8 

\Section
\Publicvar{XtNselTime}
int  selTime = 400 

\Section
\Publicvar{XtNtext}
String  text = {\langle}String{\rangle}"Hello World{\backslash}nHow are you?"

\Section
\Publicvar{XtNfg}
{\langle}XftColor{\rangle} XftColor  fg = {\langle}String{\rangle}"White"

\Section
\Publicvar{XtNbg}
{\langle}XftColor{\rangle} XftColor  bg = {\langle}String{\rangle}"Darkgreen"

\Section
\Publicvar{XtNcbg}
{\langle}XftColor{\rangle} XftColor  cbg = {\langle}String{\rangle}"Red"

\Section
\Publicvar{XtNcfg}
{\langle}XftColor{\rangle} XftColor  cfg = {\langle}String{\rangle}"Green"

\Section
\Publicvar{XtNhost}
String  host = {\langle}String{\rangle}"localhost"

\Section
\Publicvar{XtNport}
String  port = {\langle}String{\rangle}"10002"

\End\Table{Core}
XtNx&XtCX&Position &0 \cr
XtNy&XtCY&Position &0 \cr
XtNwidth&XtCWidth&Dimension &0 \cr
XtNheight&XtCHeight&Dimension &0 \cr
borderWidth&XtCBorderWidth&Dimension &0 \cr
XtNcolormap&XtCColormap&Colormap &NULL \cr
XtNdepth&XtCDepth&Int &0 \cr
destroyCallback&XtCDestroyCallback&XTCallbackList &NULL \cr
XtNsensitive&XtCSensitive&Boolean &True \cr
XtNtm&XtCTm&XTTMRec &NULL \cr
ancestorSensitive&XtCAncestorSensitive&Boolean &False \cr
accelerators&XtCAccelerators&XTTranslations &NULL \cr
borderColor&XtCBorderColor&Pixel &0 \cr
borderPixmap&XtCBorderPixmap&Pixmap &NULL \cr
background&XtCBackground&Pixel &0 \cr
backgroundPixmap&XtCBackgroundPixmap&Pixmap &NULL \cr
mappedWhenManaged&XtCMappedWhenManaged&Boolean &True \cr
XtNscreen&XtCScreen&Screen *&NULL \cr
\endTable
\Translations
\Section
\Code
{\langle}Key{\rangle}Right: forward{\underline}char() \endCode


\Section
\Code
{\langle}Key{\rangle}Left: backward{\underline}char() \endCode


\Section
\Code
{\langle}Key{\rangle}Up: prev{\underline}line() \endCode


\Section
\Code
{\langle}Key{\rangle}Down: next{\underline}line() \endCode


\Section
\Code
{\langle}Key{\rangle}Return: key{\underline}return() \endCode


\Section
\Code
{\langle}Key{\rangle}Delete: key{\underline}del() \endCode


\Section
\Code
{\langle}Key{\rangle}BackSpace: key{\underline}backspace() \endCode


\Section
\Code
{\langle}Key{\rangle}Escape: kill{\underline}child() \endCode


\Section
\Code
{\langle}Key{\rangle}: insert{\underline}char() \endCode


\End\Actions
\Section
\Action{kill{\underline}child}\Code
void kill{\underline}child({\dollar}, XEvent* event, String* params, Cardinal* num{\underline}params)
{\lbrace}
    undraw{\underline}selection({\dollar});
    te{\underline}writeln({\dollar}, "kill");
    prog{\underline}kill({\dollar});

    selection{\underline}draw({\dollar});
{\rbrace}\endCode


\Section
\Action{key{\underline}del}\Code
void key{\underline}del({\dollar}, XEvent* event, String* params, Cardinal* num{\underline}params)
{\lbrace}
    undraw{\underline}selection({\dollar});
    te{\underline}del({\dollar});
    selection{\underline}draw({\dollar});
{\rbrace}\endCode


\Section
\Action{key{\underline}backspace}\Code
void key{\underline}backspace({\dollar}, XEvent* event, String* params, Cardinal* num{\underline}params)
{\lbrace}
    undraw{\underline}selection({\dollar});
    te{\underline}left({\dollar});
    te{\underline}del({\dollar});
    selection{\underline}draw({\dollar});
{\rbrace}\endCode


\Section
\Action{key{\underline}return}\Code
void key{\underline}return({\dollar}, XEvent* event, String* params, Cardinal* num{\underline}params)
{\lbrace}
    undraw{\underline}selection({\dollar});
    exec{\underline}line({\dollar});
    te{\underline}pos1({\dollar});
    te{\underline}down({\dollar});
    selection{\underline}draw({\dollar});
{\rbrace}\endCode


\Section
\Action{insert{\underline}char}\Code
void insert{\underline}char({\dollar}, XEvent* event, String* params, Cardinal* num{\underline}params)
{\lbrace}
    int len;
    unsigned char buf[32];
    KeySym key{\underline}sim;
    uint32{\underline}t ucs4;

    len = {\underline}XawLookupString({\dollar},(XKeyEvent *) event, (char*)buf,sizeof buf, {\ampersand}key{\underline}sim );
    if (len {\langle}= 0) return;

    if(! {\dollar}INTERACTIVE {\ampersand}{\ampersand} {\dollar}child ) {\lbrace}
             dprintf({\dollar}child-{\rangle}fd[CHILD{\underline}STDIN{\underline}WR],"KEY: {\percent}*s{\backslash}n",len,buf);  
             return;
    {\rbrace}
    

    FcUtf8ToUcs4 (buf, {\ampersand}ucs4, len);
    undraw{\underline}selection({\dollar});
    te{\underline}ins({\dollar});
    te{\underline}putc({\dollar},ucs4);
    selection{\underline}draw({\dollar});
{\rbrace}\endCode


\Section
\Action{prev{\underline}line}\Code
void prev{\underline}line({\dollar}, XEvent* event, String* params, Cardinal* num{\underline}params)
{\lbrace}
    if( {\dollar}rsel.y == 0 ) return;
    {\dollar}rsel.y--;
    selection{\underline}draw({\dollar});
{\rbrace}\endCode


\Section
\Action{next{\underline}line}\Code
void next{\underline}line({\dollar}, XEvent* event, String* params, Cardinal* num{\underline}params)
{\lbrace}
    undraw{\underline}selection({\dollar});
    te{\underline}down({\dollar});
    // if( {\dollar}rsel.y {\langle} ({\dollar}gHeight-1) )
    // {\dollar}rsel.y++;
    selection{\underline}draw({\dollar});
{\rbrace}\endCode


\Section
\Action{backward{\underline}char}\Code
void backward{\underline}char({\dollar}, XEvent* event, String* params, Cardinal* num{\underline}params)
{\lbrace}
    if( {\dollar}rsel.x == 0 ) return;
    {\dollar}rsel.x--;
    selection{\underline}draw({\dollar});
{\rbrace}\endCode


\Section
\Action{forward{\underline}char}\Code
void forward{\underline}char({\dollar}, XEvent* event, String* params, Cardinal* num{\underline}params)
{\lbrace}
    if( {\dollar}rsel.x {\langle} ({\dollar}gWidth-1) )
        {\dollar}rsel.x ++;
    selection{\underline}draw({\dollar});
{\rbrace}\endCode


\End\Imports
\Section
\Code
{\incl} {\langle}stdint.h{\rangle}\endCode


\Section
\Code
{\incl} "mls.h"\endCode


\Section
\Code
{\incl} "xutil.h"\endCode


\Section
\Code
{\incl} "micro{\underline}vars.h"\endCode


\Section
\Code
{\incl} "converters-xft.h"\endCode


\Section
\Code
{\incl} "subshell.h"\endCode


\Section
\Code
{\incl} "srvconnection.h"\endCode


\Section
\Code
{\incl} "command-parser.h"\endCode


\Section
\Code
{\incl} {\langle}X11/Xaw/XawImP.h{\rangle}\endCode


\End\Privatevars
\Section
\Code
XtIntervalId  timerid\endCode


\Section
\Code
Boolean  blink\endCode


\Section
\Code
XftFont * xftFont\endCode


\Section
\Code
XftDraw * draw\endCode


\Section
\Code
XftColor * col[4]\endCode


\Section
\Code
uint32{\underline}t * scr\endCode


\Section
\Code
int  grid{\underline}pix{\underline}width\endCode


\Section
\Code
int  grid{\underline}pix{\underline}height\endCode


\Section
\Code
Bool  selection{\underline}visible\endCode


\Section
\Code
XRectangle  rsel\endCode


\Section
\Code
XRectangle  rsel{\underline}old\endCode


\Section
\Code
GC  gc{\underline}copy\endCode


\Section
\Code
uint8{\underline}t * cmd\endCode


\Section
\Code
uint  cmd{\underline}len\endCode


\Section
\Code
int  progm\endCode


\Section
\Code
struct  fork2{\underline}info * child\endCode


\Section
\Code
XtInputId  cid\endCode


\Section
\Code
XtInputId  eid\endCode


\Section
\Code
Bool  INTERACTIVE\endCode


\Section
\Code
int  server{\underline}id\endCode


\End\Methods
\Section
\Code
class{\underline}initialize()
{\lbrace} converters{\underline}xft{\underline}init(); {\rbrace}\endCode


\Section
\Code
initialize(Widget  request, {\dollar}, ArgList  args, Cardinal * num{\underline}args)
{\lbrace}
    FcChar32 bat = 0x42;
    XGlyphInfo extents;
    Display *dpy = XtDisplay({\dollar});
    font{\underline}load({\dollar},{\dollar}size);
    XftTextExtents32(dpy, {\dollar}xftFont, {\ampersand}bat, 1, {\ampersand}extents);
    {\dollar}height = ({\dollar}grid{\underline}pix{\underline}height = {\dollar}xftFont-{\rangle}ascent + {\dollar}xftFont-{\rangle}descent) * {\dollar}gHeight;
    {\dollar}width  = ({\dollar}grid{\underline}pix{\underline}width = extents.width) * {\dollar}gWidth;
    {\dollar}draw=0;
    {\dollar}cmd=0;
    {\dollar}scr=0;
    {\dollar}child=0;
    {\dollar}col[0] = {\ampersand} {\dollar}fg;
    {\dollar}col[1] = {\ampersand} {\dollar}bg;
    {\dollar}col[2] = {\ampersand} {\dollar}cfg;
    {\dollar}col[3] = {\ampersand} {\dollar}cbg;
    {\dollar}selection{\underline}visible = False;
    {\dollar}timerid = 0;
    {\dollar}gc{\underline}copy =  XtGetGC({\dollar}, 0,0 );
    {\dollar}progm = m{\underline}create(100,sizeof(char*));
    do{\underline}resize({\dollar});
    {\dollar}INTERACTIVE=1;


    cp{\underline}init();
    cp{\underline}add( "EXIT", comm{\underline}exit{\underline}cb );
    cp{\underline}add( "RUN", comm{\underline}run{\underline}cb );
    
    {\dollar}server{\underline}id = srvc{\underline}connect({\dollar},{\dollar}host,{\dollar}port, comm{\underline}recv{\underline}cb, {\dollar});
{\rbrace}\endCode


\Section
\Code
destroy({\dollar})
{\lbrace}
        cp{\underline}destroy();
        
    if( {\dollar}draw) XftDrawDestroy( {\dollar}draw );
    {\dollar}draw=0;
    font{\underline}free({\dollar});
    if( {\dollar}scr ) {\lbrace} free({\dollar}scr); {\dollar}scr=0; {\rbrace}
    if( {\dollar}cmd ) {\lbrace} free({\dollar}cmd); {\dollar}cmd=0; {\rbrace}
    if( {\dollar}progm ) {\lbrace} m{\underline}free{\underline}strings( {\dollar}progm, 0); {\dollar}progm=0; {\rbrace}
    if( {\dollar}child ) prog{\underline}kill({\dollar});

{\rbrace}\endCode


\Section
\Code
Boolean  set{\underline}values(Widget  old, Widget  request, {\dollar}, ArgList  args, Cardinal * num{\underline}args)
{\lbrace}
    if( {\dollar}old{\dollar}size != {\dollar}size ) {\lbrace}
        font{\underline}free({\dollar});
        font{\underline}load({\dollar},{\dollar}size);
        {\dollar}resize( {\dollar} );
    {\rbrace}
    return True; /* call expose */
{\rbrace}\endCode


\Section
\Code
resize({\dollar})
{\lbrace}
    {\dollar}gWidth  = {\dollar}width / {\dollar}grid{\underline}pix{\underline}width;
    {\dollar}gHeight = {\dollar}height / {\dollar}grid{\underline}pix{\underline}height;
    do{\underline}resize({\dollar});
{\rbrace}\endCode


\Section
\Code
expose({\dollar}, XEvent * event, Region  region)
{\lbrace}
    Display *dpy = XtDisplay({\dollar});
    if(!{\dollar}draw )
        {\dollar}draw = XftDrawCreate(dpy, XtWindow({\dollar}),
                              DefaultVisual(dpy, DefaultScreen(dpy)), None);
    test{\underline}convert({\dollar});
    full{\underline}draw({\dollar});
    selection{\underline}draw({\dollar});
    /*   FcChar32 bat = 0xf240 + {\dollar}status;
        XftDrawString32({\dollar}draw, {\dollar}col[{\dollar}status],{\dollar}xftFont,0,{\dollar}xftFont-{\rangle}ascent,{\ampersand}bat,1);
    */
    printf("w WxH={\percent}dx{\percent}d{\backslash}n",{\dollar}width, {\dollar}height );
{\rbrace}\endCode


\End\Utilities
\Section
\Code
font{\underline}free({\dollar})
{\lbrace}
    Display *dpy = XtDisplay({\dollar});
    if( {\dollar}xftFont ) XftFontClose(dpy, {\dollar}xftFont );
    {\dollar}xftFont=0;
{\rbrace}\endCode


\Section
\Code
font{\underline}load({\dollar}, int  size)
{\lbrace}
    Display *dpy = XtDisplay({\dollar});
    int screen = DefaultScreen(dpy);
    char fontname[30];
    sprintf(fontname, "Source Code Pro-{\percent}d", {\dollar}size );
    font{\underline}free({\dollar});
    {\dollar}xftFont = XftFontOpenName(dpy,screen, fontname );
{\rbrace}\endCode


\Section
\Code
draw{\underline}glyph{\underline}color({\dollar}, int  x, int  y, int  fg, int  bg, uint32{\underline}t  ucs4)
{\lbrace}
    int x0,y0,w,h;
    uint32{\underline}t glyph;

    x0 = x * {\dollar}grid{\underline}pix{\underline}width;
    y0 = y * {\dollar}grid{\underline}pix{\underline}height;
    w = {\dollar}grid{\underline}pix{\underline}width;
    h = {\dollar}grid{\underline}pix{\underline}height;

    if( bg {\rangle}= 0 ) XftDrawRect({\dollar}draw,{\dollar}col[bg], x0,y0,w,h );
    if( ucs4 == 0 {\bar}{\bar} ucs4 == 32 ) return;

    glyph = XftCharIndex ( XtDisplay({\dollar}), {\dollar}xftFont, ucs4 );
    XftDrawGlyphs ({\dollar}draw, {\dollar}col[fg], {\dollar}xftFont,
                   x0, y0+{\dollar}xftFont-{\rangle}ascent,
                   {\ampersand}glyph, 1);
{\rbrace}\endCode


\Section
\Code
full{\underline}draw({\dollar})
{\lbrace}
    int x,y;
    XftDrawRect({\dollar}draw,{\dollar}col[1], 0,0, {\dollar}width, {\dollar}height );
    for(x=0;x{\langle}{\dollar}gWidth;x++) {\lbrace}
        for(y=0;y{\langle}{\dollar}gHeight;y++) {\lbrace}
            draw{\underline}glyph{\underline}color({\dollar},x,y,0,-1,{\dollar}scr[x + y * {\dollar}gWidth ] );
        {\rbrace}
    {\rbrace}
{\rbrace}\endCode


\Section
\Code
putcharat({\dollar}, int  x, int  y, uint32{\underline}t  ucs4)
{\lbrace}
    if( x {\langle} {\dollar}gWidth {\ampersand}{\ampersand} x {\rangle}= 0 {\ampersand}{\ampersand} y {\langle} {\dollar}gHeight {\ampersand}{\ampersand} y {\rangle}= 0 )
        {\lbrace}
            {\dollar}scr[x + y * {\dollar}gWidth ] = ucs4;
        {\rbrace}
{\rbrace}\endCode


\Section
\Code
uint32{\underline}t  getcharat({\dollar}, int  x, int  y)
{\lbrace}
    if( x {\langle} {\dollar}gWidth {\ampersand}{\ampersand} x {\rangle}= 0 {\ampersand}{\ampersand} y {\langle} {\dollar}gHeight {\ampersand}{\ampersand} y {\rangle}= 0 )
        {\lbrace}
            return {\dollar}scr[x + y * {\dollar}gWidth ];
        {\rbrace}

    return 0;
{\rbrace}\endCode


\Section
\Code
drawcharat({\dollar}, int  x, int  y, int  fg, int  bg)
{\lbrace}
    if( x {\langle} {\dollar}gWidth {\ampersand}{\ampersand} x {\rangle}= 0 {\ampersand}{\ampersand} y {\langle} {\dollar}gHeight {\ampersand}{\ampersand} y {\rangle}= 0 )
        {\lbrace}
            draw{\underline}glyph{\underline}color({\dollar},x,y,fg,bg,{\dollar}scr[x + y * {\dollar}gWidth ] );
        {\rbrace}
{\rbrace}\endCode


\Section
\Code
draw{\underline}string({\dollar}, int  x, int  y, char * str)
{\lbrace}
    int len,l;
    unsigned char *string = (void*) str;
    uint32{\underline}t ucs4;
    len = strlen( (char*)string);
    x=0; y=0;
    while(1) {\lbrace}
        l = FcUtf8ToUcs4 (string, {\ampersand}ucs4, len);
        if( l{\langle}= 0 {\bar}{\bar} len {\langle}=0 ) return;
        if( ucs4 == '{\backslash}n' ) {\lbrace} x=0; y++; {\rbrace}
        if( x {\rangle}= {\dollar}gWidth ) {\lbrace} x=0; y++; {\rbrace}
        if( y {\rangle}= {\dollar}gHeight ) {\lbrace} y=0; {\rbrace}
        if( ucs4 {\rangle}=' ' ) {\lbrace}
            putcharat({\dollar},x,y,ucs4);
            x++;
        {\rbrace}
        string += l;
        len -= l;
    {\rbrace}
{\rbrace}\endCode


\Section
\Code
test{\underline}convert({\dollar})
{\lbrace}
    draw{\underline}string({\dollar}, 0, {\dollar}gHeight / 2, {\dollar}text );
{\rbrace}\endCode


\Section
\Code
selection{\underline}timer({\dollar}, XtIntervalId * id)
{\lbrace}
    int nr = {\dollar}blink ? 2 : 0;
    if( {\dollar}selection{\underline}visible ) {\lbrace}
        drawcharat({\dollar}, {\dollar}rsel{\underline}old.x, {\dollar}rsel{\underline}old.y, nr, nr+1 );
        {\dollar}blink = ! {\dollar}blink;
        if( id==NULL {\ampersand}{\ampersand} {\dollar}timerid ) XtRemoveTimeOut({\dollar}timerid);
        {\dollar}timerid = XtAppAddTimeOut(XtWidgetToApplicationContext({\dollar}), {\dollar}selTime,
                        (void*)selection{\underline}timer, {\dollar} );
    {\rbrace}
    else {\dollar}timerid=0;
{\rbrace}\endCode


\Section
\Code
undraw{\underline}selection({\dollar})
{\lbrace}
    if( {\dollar}selection{\underline}visible ) {\lbrace}
        drawcharat({\dollar}, {\dollar}rsel{\underline}old.x, {\dollar}rsel{\underline}old.y, 0,1 );
        {\dollar}selection{\underline}visible = False;
        if({\dollar}timerid) {\lbrace}
            XtRemoveTimeOut({\dollar}timerid);
            {\dollar}timerid=0;
        {\rbrace}
    {\rbrace}
{\rbrace}\endCode


\Section
\Code
selection{\underline}draw({\dollar})
{\lbrace}
    if( {\dollar}rsel.x {\rangle}= {\dollar}gWidth ) {\dollar}rsel.x = {\dollar}gWidth-1;
    if( {\dollar}rsel.y {\rangle}= {\dollar}gHeight ) {\dollar}rsel.y = {\dollar}gHeight-1;
    undraw{\underline}selection({\dollar});
    {\dollar}selection{\underline}visible = True;
    {\dollar}rsel{\underline}old = {\dollar}rsel;
    {\dollar}blink = True; selection{\underline}timer({\dollar},0);
{\rbrace}\endCode


\Section
\Code
scr{\underline}scroll{\underline}up({\dollar})
{\lbrace}
        memcpy( {\dollar}scr, {\dollar}scr + {\dollar}gWidth,
                ({\dollar}gWidth*({\dollar}gHeight-1))*4 );
        memset( {\dollar}scr + {\dollar}gWidth*({\dollar}gHeight-1), 0, {\dollar}gWidth*4 );
{\rbrace}\endCode


\Section
\Code
pix{\underline}scroll{\underline}up({\dollar})
{\lbrace}
/*
      XCopyArea(display, src, dest, gc, src{\underline}x, src{\underline}y, width, height,  dest{\underline}x, dest{\underline}y)
      Display *display;
      Drawable src, dest;
      GC gc;
      int src{\underline}x, src{\underline}y;
      unsigned int width, height;
      int dest{\underline}x, dest{\underline}y;
*/
    XCopyArea(XtDisplay({\dollar}), XtWindow({\dollar}), XtWindow({\dollar}), DefaultGC(XtDisplay({\dollar}),0),
              0, {\dollar}grid{\underline}pix{\underline}height,
              {\dollar}grid{\underline}pix{\underline}width * ({\dollar}gWidth),
              {\dollar}grid{\underline}pix{\underline}height * ({\dollar}gHeight-1),
              0,0 );

    /* clear background */
    XftDrawRect({\dollar}draw,{\dollar}col[1],
    0,      {\dollar}grid{\underline}pix{\underline}height * ({\dollar}gHeight-1),
    {\dollar}width, {\dollar}grid{\underline}pix{\underline}height );

{\rbrace}\endCode


\Section
\Code
te{\underline}pos1({\dollar})
{\lbrace}
    {\dollar}rsel.x = 0;
{\rbrace}\endCode


\Section
\Code
te{\underline}tab({\dollar})
{\lbrace}
    int x = {\dollar}rsel.x;
    x /= 8;
    x ++;
    x *= 8;
    if( x {\langle} {\dollar}gWidth ) {\dollar}rsel.x = x;
{\rbrace}\endCode


\Section
\Code
te{\underline}down({\dollar})
{\lbrace}
    if( {\dollar}rsel.y {\langle} ({\dollar}gHeight-1) ) {\lbrace}
        {\dollar}rsel.y++;
        return;
    {\rbrace}

    pix{\underline}scroll{\underline}up({\dollar});
    scr{\underline}scroll{\underline}up({\dollar});
{\rbrace}\endCode


\Section
\Code
Bool  te{\underline}parse({\dollar}, uint32{\underline}t  ucs4)
{\lbrace}
    switch( ucs4 ) {\lbrace}
    default: break;
    case 10: te{\underline}pos1({\dollar}); te{\underline}down({\dollar}); return True;
    case '{\backslash}t': te{\underline}tab({\dollar}); return True;
    case 13: te{\underline}pos1({\dollar}); return True;
    {\rbrace}
    return False;
{\rbrace}\endCode


\Section
\Code
te{\underline}right({\dollar})
{\lbrace}
    if( {\dollar}rsel.x {\langle} ({\dollar}gWidth-1) ) {\lbrace}
        {\dollar}rsel.x++;
        return;
    {\rbrace}
    te{\underline}pos1({\dollar});
    te{\underline}down({\dollar});
{\rbrace}\endCode


\Section
\Code
pix{\underline}del({\dollar})
{\lbrace}
   int x0,y0,x1,y1,w,h,x,y;
   x = {\dollar}rsel.x; y={\dollar}rsel.y;
   /*-----------------------*/
   /* destination           */
   x0 = x;
   y0 = y;
   /*-----------------------*/
   /* width, height         */
   w = {\dollar}gWidth - (x+1);
   h = 1;
   /*-----------------------*/
   /* source                */
   x1 = x0+1;
   y1 = y0;
   /*-----------------------*/
   /* transformation        */
   w   *= {\dollar}grid{\underline}pix{\underline}width;
   x0  *= {\dollar}grid{\underline}pix{\underline}width;
   x1  *= {\dollar}grid{\underline}pix{\underline}width;
   h   *= {\dollar}grid{\underline}pix{\underline}height;
   y0  *= {\dollar}grid{\underline}pix{\underline}height;
   y1  *= {\dollar}grid{\underline}pix{\underline}height;

   XCopyArea(XtDisplay({\dollar}), XtWindow({\dollar}), XtWindow({\dollar}), DefaultGC(XtDisplay({\dollar}),0),
             x1,y1,w,h, /* source */
             x0,y0      /* destination */
             );
{\rbrace}\endCode


\Section
\Code
pix{\underline}ins({\dollar})
{\lbrace}
   int x0,y0,x1,y1,w,h,x,y;
   x = {\dollar}rsel.x; y={\dollar}rsel.y;
   /*-----------------------*/
   /* destination           */
   x0 = x+1;
   y0 = y;
   /*-----------------------*/
   /* width, height         */
   w = {\dollar}gWidth - (x+1);
   h = 1;
   /*-----------------------*/
   /* source                */
   x1 = x;
   y1 = y;
   /*-----------------------*/
   /* transformation        */
   w   *= {\dollar}grid{\underline}pix{\underline}width;
   x0  *= {\dollar}grid{\underline}pix{\underline}width;
   x1  *= {\dollar}grid{\underline}pix{\underline}width;
   h   *= {\dollar}grid{\underline}pix{\underline}height;
   y0  *= {\dollar}grid{\underline}pix{\underline}height;
   y1  *= {\dollar}grid{\underline}pix{\underline}height;

   XCopyArea(XtDisplay({\dollar}), XtWindow({\dollar}), XtWindow({\dollar}), DefaultGC(XtDisplay({\dollar}),0),
             x1,y1,w,h, /* source */
             x0,y0      /* destination */
             );
{\rbrace}\endCode


\Section
\Code
te{\underline}ins({\dollar})
{\lbrace}
   int i,x,y;
   x = {\dollar}rsel.x; y={\dollar}rsel.y;
   uint32{\underline}t *s = {\dollar}scr+y* {\dollar}gWidth;

   /* letztes zeichen der zeile mit {\langle}null{\rangle} füllen */
   if( x == {\dollar}gWidth -1 ) {\lbrace}
      s[x] = 0;
      return;
   {\rbrace}

   for(i={\dollar}gWidth-2; i{\rangle}=x ; i-- )
              s[i+1] = s[i];
   pix{\underline}ins({\dollar});
   s[x] = 32;
{\rbrace}\endCode


\Section
\Code
te{\underline}del({\dollar})
{\lbrace}
   uint16{\underline}t i,x,y;
   x = {\dollar}rsel.x; y={\dollar}rsel.y;
   uint32{\underline}t *s = {\dollar}scr+y* {\dollar}gWidth;

   /* letztes zeichen der zeile mit {\langle}null{\rangle} füllen */
   if( x == {\dollar}gWidth -1 ) {\lbrace}
      s[x] = 0;
      return;
   {\rbrace}

   for(i=x+1; i{\langle} {\dollar}gWidth; i++ )
              s[i-1] = s[i];
   pix{\underline}del({\dollar});
{\rbrace}\endCode


\Section
\Code
te{\underline}putc({\dollar}, uint32{\underline}t  ucs4)
{\lbrace}
    if( {\dollar}rsel.x {\rangle}= {\dollar}gWidth ) {\dollar}rsel.x = {\dollar}gWidth-1;
    if( {\dollar}rsel.y {\rangle}= {\dollar}gHeight ) {\dollar}rsel.y = {\dollar}gHeight-1;
    if( te{\underline}parse({\dollar},ucs4) ) return;
    putcharat({\dollar}, {\dollar}rsel.x, {\dollar}rsel.y,ucs4);
    drawcharat({\dollar}, {\dollar}rsel.x, {\dollar}rsel.y, 0,1 );
    {\dollar}selection{\underline}visible = False;
    te{\underline}right({\dollar});
{\rbrace}\endCode


\Section
\Code
te{\underline}left({\dollar})
{\lbrace}
    if( {\dollar}rsel.x == 0 ) return;
    {\dollar}rsel.x--;
{\rbrace}\endCode


\Section
\Code
exec{\underline}line({\dollar})
{\lbrace}
    int i,len, space{\underline}left;
    unsigned char *dest;
    // char *string = malloc( {\dollar}gWidth * 6 + 2 );
    char *string = (void*){\dollar}cmd;
    uint32{\underline}t *s = {\dollar}scr + {\dollar}rsel.y * {\dollar}gWidth;
    dest = (void*)string;
    space{\underline}left = {\dollar}cmd{\underline}len;
    for( i=0; i{\langle} {\dollar}gWidth; i++ )
        {\lbrace}
            if( space{\underline}left {\langle} 7 ) break;
            len = FcUcs4ToUtf8( *s++, dest );
            dest+=len;
            space{\underline}left -= len;
        {\rbrace}
    *dest = 0;
    printf("str: {\percent}s{\backslash}n", string );

    /* check if there is a line number prefix */
    if(! prog{\underline}insert{\underline}line({\dollar}, string) ) return;

    if( strncmp( string, "list", 4 ) == 0 ) prog{\underline}list({\dollar});
    else if( strncmp( string, "save", 4 ) == 0 ) prog{\underline}save({\dollar});
    else if( strncmp( string, "load", 4 ) == 0 ) prog{\underline}load({\dollar});
    else if( strncmp( string, "run", 3 ) == 0 ) prog{\underline}run({\dollar});
    else if( strncmp( string, "parse", 5 ) == 0 ) prog{\underline}parse({\dollar});
    else if( {\dollar}child ) prog{\underline}interpret({\dollar});
{\rbrace}\endCode


\Section
\Code
prog{\underline}interpret({\dollar})
{\lbrace}
    char *s = (void*){\dollar}cmd;
    while(*s {\ampersand}{\ampersand} *s == '{\rangle}' ) s++;
    dprintf( {\dollar}child-{\rangle}fd[3], "{\percent}s{\backslash}n", s );
{\rbrace}\endCode


\Section
\Code
prog{\underline}child{\underline}pipe{\underline}read({\dollar}, int  pipe)
{\lbrace}
   if( ! {\dollar}child ) ERR("memory corruption");

   if( fork2{\underline}read({\dollar}child, pipe) ) {\lbrace}     
        prog{\underline}kill({\dollar});
        return;
   {\rbrace}

   int buf = m{\underline}create(50,1);    

   undraw{\underline}selection({\dollar});
   
   while(! fork2{\underline}getline({\dollar}child, pipe, buf ) ) {\lbrace}

      te{\underline}writeln({\dollar}, m{\underline}buf(buf));
      
      if( strncmp(m{\underline}buf(buf),"RUN",3)==0 ) {\lbrace}
          {\dollar}INTERACTIVE=0;
          fprintf(stderr, "interactive mode off{\backslash}n" );
      {\rbrace}
      else if( strncmp(m{\underline}buf(buf),"EXIT",4)==0 ) {\lbrace}
          fprintf(stderr, "interactive mode ON{\backslash}n" );
          {\dollar}INTERACTIVE=1;
      {\rbrace}
   {\rbrace}
   selection{\underline}draw({\dollar});

{\rbrace}\endCode


\Section
\Code
prog{\underline}child{\underline}read{\underline}stdout({\dollar}, int * source, XtInputId * id)
{\lbrace}
    prog{\underline}child{\underline}pipe{\underline}read({\dollar}, CHILD{\underline}STDOUT{\underline}RD );
{\rbrace}\endCode


\Section
\Code
prog{\underline}child{\underline}read{\underline}stderr({\dollar}, int * source, XtInputId * id)
{\lbrace}
    prog{\underline}child{\underline}pipe{\underline}read({\dollar}, CHILD{\underline}STDERR{\underline}RD );
{\rbrace}\endCode


\Section
\Code
prog{\underline}run({\dollar})
{\lbrace}
        if( {\dollar}child ) prog{\underline}kill({\dollar});
        {\dollar}child = fork2{\underline}open( "./interpreter.sh", "-i", NULL, "-e", "{\underline}PROMPT=''", NULL );
        {\dollar}cid = XtAppAddInput( XtWidgetToApplicationContext({\dollar}),
                       {\dollar}child-{\rangle}fd[CHILD{\underline}STDOUT{\underline}RD],
                       (XtPointer)XtInputReadMask,
                       (XtInputCallbackProc)prog{\underline}child{\underline}read{\underline}stdout,
                       {\dollar} );

        {\dollar}eid = XtAppAddInput( XtWidgetToApplicationContext({\dollar}),
                       {\dollar}child-{\rangle}fd[CHILD{\underline}STDERR{\underline}RD],
                       (XtPointer)XtInputReadMask,
                       (XtInputCallbackProc)prog{\underline}child{\underline}read{\underline}stderr,
                       {\dollar} );

         if( {\dollar}child ) fork2{\underline}write({\dollar}child, "hello world{\backslash}n");
{\rbrace}\endCode


\Section
\Code
char * progm{\underline}alloc({\dollar}, uint32{\underline}t  ln, char * s)
{\lbrace}
        char *buf = malloc( 5 + strlen(s) );
        *(uint32{\underline}t *)buf = ln;
        while( isspace(*s) ) s++;
        strcpy( buf +4, s );
        return buf;
{\rbrace}\endCode


\Section
\Code
progm{\underline}insert({\dollar}, int  i, uint32{\underline}t  arg, char * s)
{\lbrace}
        m{\underline}ins({\dollar}progm,i, 1);
        STR({\dollar}progm, i) = progm{\underline}alloc({\dollar},arg,s);
{\rbrace}\endCode


\Section
\Code
progm{\underline}overwrite({\dollar}, int  i, uint32{\underline}t  arg, char * s)
{\lbrace}
        free( STR({\dollar}progm,i));
        STR({\dollar}progm, i) = progm{\underline}alloc({\dollar},arg,s);
{\rbrace}\endCode


\Section
\Code
progm{\underline}append({\dollar}, uint32{\underline}t  arg, char * s)
{\lbrace}
        char *buf = progm{\underline}alloc({\dollar},arg,s);
        m{\underline}put( {\dollar}progm, {\ampersand}buf );
{\rbrace}\endCode


\Section
finde zeile {\tt num} im puffer
  returns: match
  {\tt *pos} == zeilennummer
  match == 0 : gefunden
  match <  0 : num größte zeile
  match >  1 : *pos größer als {\tt num}


\Code
int  progm{\underline}find({\dollar}, uint32{\underline}t  num, int * pos)
{\lbrace}
        int i;
        uint32{\underline}t n;
        char *s;

        for(i=0;i{\langle}m{\underline}len({\dollar}progm);i++) {\lbrace}
          s= STR({\dollar}progm,i);
          n = *(uint32{\underline}t *) s;
          if( n == num ) {\lbrace} *pos=i; return 0; {\rbrace}
          else if( n {\rangle} num ) {\lbrace} *pos=i; return 1; {\rbrace};
        {\rbrace}
        *pos = i;
        return -1;
{\rbrace}\endCode


\Section
\Code
progm{\underline}put({\dollar}, uint32{\underline}t  arg, char * ln)
{\lbrace}
        int i;
        uint32{\underline}t n;
        char *s;

        for(i=0;i{\langle}m{\underline}len({\dollar}progm);i++) {\lbrace}
          s= STR({\dollar}progm,i);
          n = *(uint32{\underline}t *) s;
          if( n == arg ) {\lbrace} progm{\underline}overwrite( {\dollar}, i, arg, ln ); return; {\rbrace}
          else if( n {\rangle} arg ) {\lbrace} progm{\underline}insert({\dollar}, i, arg, ln ); return; {\rbrace}
        {\rbrace}
        progm{\underline}append({\dollar},arg, ln );
{\rbrace}\endCode


\Section
\Code
Bool  prog{\underline}insert{\underline}line({\dollar}, char * buf)
{\lbrace}
        char *s = (void*) buf;
        char *endp;
        int arg;

        arg = strtol( s,{\ampersand}endp,10);
        if( s == endp ) {\lbrace} return 1; {\rbrace}

        progm{\underline}put({\dollar}, arg, endp );
        return 0;
{\rbrace}\endCode


\Section
load a program file to memory
  syntax: load filename




\Code
prog{\underline}error({\dollar}, char * s)
{\lbrace}
        te{\underline}down({\dollar});
        te{\underline}pos1({\dollar});
        te{\underline}write({\dollar},s);
        te{\underline}down({\dollar});
        te{\underline}pos1({\dollar});
{\rbrace}\endCode


\Section
\Code
prog{\underline}load({\dollar})
{\lbrace}
        char *s = (void*){\dollar}cmd+5;
        int p = m{\underline}create(20,1);

        while( *s {\ampersand}{\ampersand} ! isspace(*s) )
               m{\underline}putc( p, *s++ );
        m{\underline}putc(p,0);

        FILE *fp = fopen( m{\underline}buf(p), "r" );
        if( !fp ) {\lbrace} prog{\underline}error( {\dollar}, "file read error" ); goto leave; {\rbrace}
        m{\underline}clear(p);


        while( m{\underline}fscan2( p, '{\backslash}n', fp ) == '{\backslash}n' ) {\lbrace}
                prog{\underline}insert{\underline}line({\dollar},m{\underline}buf(p));
                m{\underline}clear(p);
        {\rbrace}

        fclose(fp);
        leave: m{\underline}free(p);

{\rbrace}\endCode


\Section
save a program file to filesystem
  syntax: save filename


\Code
prog{\underline}save({\dollar})
{\lbrace}
        /* get two integer args */
        char *s = (void*){\dollar}cmd+5;
        int p = m{\underline}create(20,1);
        int i;

        while( *s {\ampersand}{\ampersand} ! isspace(*s) )
               m{\underline}putc( p, *s++ );
        m{\underline}putc(p,0);

        FILE *fp = fopen( m{\underline}buf(p), "w" );
        if( !fp ) {\lbrace} prog{\underline}error( {\dollar}, "file write error" ); goto leave; {\rbrace}

        for(i=0;i{\langle}m{\underline}len({\dollar}progm);i++)
        {\lbrace}
                s = STR({\dollar}progm,i);
                uint32{\underline}t n = *(uint32{\underline}t *)s; s+=4;
                fprintf(fp,"{\percent}u {\percent}s{\backslash}n", n,s );
        {\rbrace}

        fclose(fp);
        leave: m{\underline}free(p);
{\rbrace}\endCode


\Section
\Code
prog{\underline}parse({\dollar})
{\lbrace}

    if(! {\dollar}child ) {\lbrace}
        prog{\underline}error({\dollar},"starting subshell with 'run'");
        prog{\underline}run({\dollar});
    {\rbrace}

    int from,len;

    from = 0;
    len  = m{\underline}len({\dollar}progm);

    /* get two integer args */
    char *s = (void*){\dollar}cmd+6;
    char *endp;
    from = strtol( s,{\ampersand}endp,10);
    if( s == endp ) {\lbrace} from=0; goto cont; {\rbrace}

 cont:
    prog{\underline}do{\underline}parse({\dollar}, from, len);
{\rbrace}\endCode


\Section
\Code
prog{\underline}list({\dollar})
{\lbrace}
        int arg,from,len;

        from = 0;
        len  = 10;

        /* get two integer args */
        char *s = (void*){\dollar}cmd+5;
        char *endp;
        arg = strtol( s,{\ampersand}endp,10);
        if( s == endp ) {\lbrace} arg=0; goto cont; {\rbrace}

        s=endp; if( *s == ',' ) s++;
        from = arg;
        arg = strtol( s,{\ampersand}endp,10);
        if( s == endp ) {\lbrace} arg=0; goto cont; {\rbrace}
        len = arg;

        cont:
        te{\underline}pos1({\dollar}); te{\underline}down({\dollar});
        prog{\underline}do{\underline}list({\dollar}, from, len);
{\rbrace}\endCode


\Section
\Code
te{\underline}write({\dollar}, char * str)
{\lbrace}
    int len,l;
    unsigned char *string = (void*) str;
    uint32{\underline}t ucs4;
    len = strlen( (char*)string);

    while(1) {\lbrace}
        l = FcUtf8ToUcs4 (string, {\ampersand}ucs4, len);
        if( l{\langle}= 0 {\bar}{\bar} len {\langle}=0 ) return;
        te{\underline}putc({\dollar},ucs4);
        string += l;
        len -= l;
    {\rbrace}
{\rbrace}\endCode


\Section
\Code
te{\underline}writeln({\dollar}, char * s)
{\lbrace}
        te{\underline}write({\dollar},s);
        te{\underline}pos1({\dollar});
        te{\underline}down({\dollar});
{\rbrace}\endCode


\Section
\Code
prog{\underline}print({\dollar}, int  i)
{\lbrace}
        char buf[20];


        char *s = STR({\dollar}progm,i);
        uint32{\underline}t num = *(uint32{\underline}t *)s;
        s+=4;

        sprintf(buf,"{\percent}d", num);
        te{\underline}write({\dollar},buf);
        te{\underline}putc({\dollar},32);
        te{\underline}write({\dollar},s);
{\rbrace}\endCode


\Section
\Code
prog{\underline}do{\underline}list({\dollar}, int  from, int  len)
{\lbrace}
        int i;
        int line;

        int match  = progm{\underline}find( {\dollar}, from, {\ampersand}line );
        if( match {\langle} 0 ) return;

        for(i=line;i{\langle}line+len;i++) {\lbrace}
                if( i {\rangle}= m{\underline}len({\dollar}progm) ) break;
                prog{\underline}print({\dollar},i);
                te{\underline}pos1({\dollar});
                te{\underline}down({\dollar});
        {\rbrace}
{\rbrace}\endCode


\Section
\Code
prog{\underline}do{\underline}parse({\dollar}, int  from, int  len)
{\lbrace}
        int i;
        int line;

        int match  = progm{\underline}find( {\dollar}, from, {\ampersand}line );
        if( match {\langle} 0 ) return;

        for(i=line;i{\langle}line+len;i++) {\lbrace}
                if( i {\rangle}= m{\underline}len({\dollar}progm) ) break;
                char *s = STR({\dollar}progm,i);
                s+=4;
                dprintf( {\dollar}child-{\rangle}fd[3], "{\percent}s{\backslash}n", s );
        {\rbrace}
{\rbrace}\endCode


\Section
\Code
prog{\underline}kill({\dollar})
{\lbrace}
        if( {\dollar}child ) {\lbrace}
            fork2{\underline}close( {\dollar}child );
            {\dollar}child=0;
        {\rbrace}
        if( {\dollar}cid ) {\lbrace} XtRemoveInput({\dollar}cid); {\dollar}cid=0; {\rbrace}
        if( {\dollar}eid ) {\lbrace} XtRemoveInput({\dollar}eid); {\dollar}eid=0; {\rbrace}
        {\dollar}INTERACTIVE=1;
{\rbrace}\endCode


\Section
\Code
do{\underline}resize({\dollar})
{\lbrace}
    {\dollar}scr=realloc({\dollar}scr, {\dollar}gWidth * {\dollar}gHeight * 4 );
    memset({\dollar}scr,0,{\dollar}gWidth * {\dollar}gHeight * 4 );
    {\dollar}cmd=realloc({\dollar}cmd,{\dollar}cmd{\underline}len = ({\dollar}gWidth+2) * 6 );
    printf("WxH={\percent}dx{\percent}d{\backslash}n",{\dollar}gWidth, {\dollar}gHeight );
    printf("pix WxH={\percent}dx{\percent}d{\backslash}n",{\dollar}grid{\underline}pix{\underline}width, {\dollar}grid{\underline}pix{\underline}height );
    printf("w WxH={\percent}dx{\percent}d{\backslash}n",{\dollar}width, {\dollar}height );
{\rbrace}\endCode


\Section
\Code
comm{\underline}exit{\underline}cb({\dollar})
{\lbrace}
        TRACE(2,"exit");
        {\dollar}INTERACTIVE=0;
{\rbrace}\endCode


\Section
\Code
comm{\underline}run{\underline}cb({\dollar})
{\lbrace}
        TRACE(2,"run");
        {\dollar}INTERACTIVE=1;
{\rbrace}\endCode


\Section
\Code
comm{\underline}recv{\underline}cb({\dollar}, void * u, void * d)
{\lbrace}
    TRACE(2,"read line");       
    int line = (intptr{\underline}t) d;
    if( line {\langle} 1 ) return;
    cp{\underline}func{\underline}t func = cp{\underline}lookup(line);
    if( func ) func({\dollar});
{\rbrace}\endCode


\End\bye
