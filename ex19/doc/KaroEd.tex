\input wbuildmac.tex
\Class{KaroEd}\Publicvars
\Table{KaroEd}
XtNlocked&XtCLocked&Bool &0 \cr
XtNlines&XtCLines&StringMArray &0 \cr
XtNmin{\underline}lines&XtCMin{\underline}lines&int &1 \cr
XtNgrid{\underline}width&XtCGrid{\underline}width&int &1 \cr
XtNgrid{\underline}height&XtCGrid{\underline}height&int &1 \cr
XtNauto{\underline}resize&XtCAuto{\underline}resize&int &1 \cr
\endTable
\Section
\Publicvar{XtNlocked}
Bool  locked = 0 

\Section
\Publicvar{XtNlines}
StringMArray  lines = 0 

\Section
\Publicvar{XtNmin{\underline}lines}
int  min{\underline}lines = 1 

\Section
\Publicvar{XtNgrid{\underline}width}
int  grid{\underline}width = 1 

\Section
\Publicvar{XtNgrid{\underline}height}
int  grid{\underline}height = 1 

\Section
\Publicvar{XtNauto{\underline}resize}
int  auto{\underline}resize = 1 

\End\Table{Wheel}
XtNxftFont&XtCXFtFont&XftFont&"Sans-22"\cr
XtNcallback&XtCCallback&Callback&NULL \cr
XtNbg{\underline}norm&XtCBg{\underline}norm&Pixel&"lightblue"\cr
XtNbg{\underline}sel&XtCBg{\underline}sel&Pixel&"yellow"\cr
XtNbg{\underline}hi&XtCBg{\underline}hi&Pixel&"red"\cr
XtNfg{\underline}norm&XtCFg{\underline}norm&Pixel&"black"\cr
XtNfg{\underline}sel&XtCFg{\underline}sel&Pixel&"green"\cr
XtNfg{\underline}hi&XtCFg{\underline}hi&Pixel&"white"\cr
XtNuser{\underline}data&XtCUser{\underline}data&Int &0 \cr
XtNfocus{\underline}group&XtCFocus{\underline}group&String &""\cr
XtNstate&XtCState&Int &0 \cr
XtNregister{\underline}focus{\underline}group&XtCRegister{\underline}focus{\underline}group&Boolean &True \cr
\endTable
\Table{Core}
XtNx&XtCX&Position &0 \cr
XtNy&XtCY&Position &0 \cr
XtNwidth&XtCWidth&Dimension &0 \cr
XtNheight&XtCHeight&Dimension &0 \cr
borderWidth&XtCBorderWidth&Dimension &0 \cr
XtNcolormap&XtCColormap&Colormap &NULL \cr
XtNdepth&XtCDepth&Int &0 \cr
destroyCallback&XtCDestroyCallback&XTCallbackList &NULL \cr
XtNsensitive&XtCSensitive&Boolean &True \cr
XtNtm&XtCTm&XTTMRec &NULL \cr
ancestorSensitive&XtCAncestorSensitive&Boolean &False \cr
accelerators&XtCAccelerators&XTTranslations &NULL \cr
borderColor&XtCBorderColor&Pixel &0 \cr
borderPixmap&XtCBorderPixmap&Pixmap &NULL \cr
background&XtCBackground&Pixel &0 \cr
backgroundPixmap&XtCBackgroundPixmap&Pixmap &NULL \cr
mappedWhenManaged&XtCMappedWhenManaged&Boolean &True \cr
XtNscreen&XtCScreen&Screen *&NULL \cr
\endTable
\Translations
\Section
\Code
{\langle}EnterWindow{\rangle}: highlight() \endCode


\Section
\Code
{\langle}LeaveWindow{\rangle}: reset() \endCode


\Section
\Code
{\langle}Btn2Down{\rangle}: info() \endCode


\Section
\Code
{\langle}Btn1Down{\rangle}: selection{\underline}set() \endCode


\Section
\Code
{\langle}Btn1Motion{\rangle}: motion{\underline}start() \endCode


\Section
\Code
{\langle}Btn1Up{\rangle}: selection{\underline}end() \endCode


\Section
\Code
{\langle}Key{\rangle}Right: forward{\underline}char() \endCode


\Section
\Code
{\langle}Key{\rangle}Left: backward{\underline}char() \endCode


\Section
\Code
{\langle}Key{\rangle}Up: prev{\underline}line() \endCode


\Section
\Code
{\langle}Key{\rangle}Down: next{\underline}line() \endCode


\Section
\Code
{\langle}Key{\rangle}Delete: remove{\underline}char() \endCode


\Section
\Code
{\langle}Key{\rangle}BackSpace: backward{\underline}char() remove{\underline}char() \endCode


\Section
\Code
{\langle}Key{\rangle}Return: key{\underline}return() \endCode


\Section
\Code
{\langle}Ctrl{\rangle}v: insert{\underline}selection() \endCode


\Section
\Code
{\langle}Key{\rangle}: insert{\underline}char() \endCode


\End\Actions
\Section
\Action{copy{\underline}selection}\Code
void copy{\underline}selection({\dollar}, XEvent* event, String* params, Cardinal* num{\underline}params)
{\lbrace}
    /*
    XStoreBuffer(display, bytes, nbytes, buffer)
      Display *display;
      char *bytes;
      int nbytes;
      int buffer;
    */


{\rbrace}\endCode


\Section
\Action{insert{\underline}selection}\Code
void insert{\underline}selection({\dollar}, XEvent* event, String* params, Cardinal* num{\underline}params)
{\lbrace}
    XtGetSelectionValue({\dollar}, XA{\underline}PRIMARY, {\dollar}XA{\underline}UTF8{\underline}STRING,
                        RequestSelection,
                        NULL, 0 );
{\rbrace}\endCode


\Section
\Action{key{\underline}return}\Code
void key{\underline}return({\dollar}, XEvent* event, String* params, Cardinal* num{\underline}params)
{\lbrace}
    Bool d = split{\underline}line{\underline}at{\underline}cursor({\dollar}) {\bar} cursor{\underline}down({\dollar}) {\bar} cursor{\underline}pos1({\dollar});
    if( d )  redraw{\underline}full({\dollar});
    selection{\underline}draw({\dollar});
{\rbrace}\endCode


\Section
\Action{prev{\underline}line}\Code
void prev{\underline}line({\dollar}, XEvent* event, String* params, Cardinal* num{\underline}params)
{\lbrace}
    if( {\dollar}rsel.y == 0 ) return;
    {\dollar}rsel.y--;
    if( {\dollar}rsel.y {\langle} {\dollar}top{\underline}y ) {\lbrace}
        {\dollar}top{\underline}y = {\dollar}rsel.y;
        redraw{\underline}full({\dollar});
    {\rbrace}
    selection{\underline}draw({\dollar});
{\rbrace}\endCode


\Section
\Action{next{\underline}line}\Code
void next{\underline}line({\dollar}, XEvent* event, String* params, Cardinal* num{\underline}params)
{\lbrace}
    {\dollar}rsel.y++;
    if( {\dollar}rsel.y {\rangle}= {\dollar}grid{\underline}height ) {\lbrace}
        {\dollar}top{\underline}y = {\dollar}rsel.y - {\dollar}grid{\underline}height + 1;
        redraw{\underline}full({\dollar});
    {\rbrace}
    selection{\underline}draw({\dollar});
{\rbrace}\endCode


\Section
\Action{backward{\underline}char}\Code
void backward{\underline}char({\dollar}, XEvent* event, String* params, Cardinal* num{\underline}params)
{\lbrace}
    if( {\dollar}rsel.x == 0 ) return;
    {\dollar}rsel.x--;
    if( {\dollar}rsel.x {\langle} {\dollar}top{\underline}x )
    {\lbrace}
        {\dollar}top{\underline}x = {\dollar}rsel.x;
        redraw{\underline}full({\dollar});
    {\rbrace}
    selection{\underline}draw({\dollar});
{\rbrace}\endCode


\Section
\Action{forward{\underline}char}\Code
void forward{\underline}char({\dollar}, XEvent* event, String* params, Cardinal* num{\underline}params)
{\lbrace}
    cursor{\underline}right({\dollar});
{\rbrace}\endCode


\Section
\Action{info}\Code
void info({\dollar}, XEvent* event, String* params, Cardinal* num{\underline}params)
{\lbrace}
    int i;
    for(i=0;i{\langle}m{\underline}len({\dollar}lines);i++) {\lbrace}

        printf("'{\percent}s'{\backslash}n", STR({\dollar}lines,i));
    {\rbrace}
    printf("{\backslash}n");
{\rbrace}\endCode


\Section
\Action{remove{\underline}char}\Code
void remove{\underline}char({\dollar}, XEvent* event, String* params, Cardinal* num{\underline}params)
{\lbrace}
    /* could be expanded to remove region */
    if( current{\underline}line{\underline}empty({\dollar}) ) {\lbrace}
        remove{\underline}current{\underline}line({\dollar});
        redraw{\underline}full({\dollar});
    {\rbrace}
    else {\lbrace}
        remove{\underline}char{\underline}at{\underline}cursor({\dollar});
        draw{\underline}line{\underline}from({\dollar}, {\dollar}rsel.x, {\dollar}rsel.y );
    {\rbrace}
{\rbrace}\endCode


\Section
\Action{insert{\underline}char}\Code
void insert{\underline}char({\dollar}, XEvent* event, String* params, Cardinal* num{\underline}params)
{\lbrace}
    int len;
    unsigned char buf[32];
    KeySym key{\underline}sim;

    len = {\underline}XawLookupString({\dollar},(XKeyEvent *) event, (char*)buf,sizeof buf, {\ampersand}key{\underline}sim );
    if (len {\langle}= 0) return;

    insert{\underline}char{\underline}at{\underline}cursor({\dollar}, (char*)buf, len );
    draw{\underline}line{\underline}from({\dollar}, {\dollar}rsel.x, {\dollar}rsel.y );
    cursor{\underline}right({\dollar});
{\rbrace}\endCode


\Section
\Action{selection{\underline}clear}\Code
void selection{\underline}clear({\dollar}, XEvent* event, String* params, Cardinal* num{\underline}params)
{\lbrace}
    selection{\underline}clear{\underline}a({\dollar});
{\rbrace}\endCode


\Section
\Action{selection{\underline}set}\Code
void selection{\underline}set({\dollar}, XEvent* event, String* params, Cardinal* num{\underline}params)
{\lbrace}
    int y = event-{\rangle}xbutton.y;
    int x = event-{\rangle}xbutton.x;
    coordinate{\underline}to{\underline}text({\dollar},{\ampersand}x,{\ampersand}y);
    {\dollar}rsel.x=x; {\dollar}rsel.width=1;
    {\dollar}rsel.y=y; {\dollar}rsel.height=1;
    selection{\underline}draw({\dollar});
{\rbrace}\endCode


\Section
\Action{highlight}\Code
void highlight({\dollar}, XEvent* event, String* params, Cardinal* num{\underline}params)
{\lbrace}
  if( {\dollar}locked ) return;
{\rbrace}\endCode


\Section
\Action{reset}\Code
void reset({\dollar}, XEvent* event, String* params, Cardinal* num{\underline}params)
{\lbrace}
{\rbrace}\endCode


\Section
\Action{notify}\Code
void notify({\dollar}, XEvent* event, String* params, Cardinal* num{\underline}params)
{\lbrace}
  if( {\dollar}locked ) return;
  {\dollar}value = {\dollar}value +1;
  XtVaSetValues({\dollar},"label", get{\underline}name({\dollar}), NULL );
  XtCallCallbackList( {\dollar}, {\dollar}callback, (XtPointer) {\dollar}value );
{\rbrace}\endCode


\Section
\Action{selection{\underline}end}\Code
void selection{\underline}end({\dollar}, XEvent* event, String* params, Cardinal* num{\underline}params)
{\lbrace}
    ulong t = (ulong ) event-{\rangle}xbutton.time;
    if( {\dollar}rsel.width{\rangle}1 ) own{\underline}primary{\underline}selection({\dollar},t);
{\rbrace}\endCode


\Section
\Action{motion{\underline}start}\Code
void motion{\underline}start({\dollar}, XEvent* event, String* params, Cardinal* num{\underline}params)
{\lbrace}
  if( {\dollar}locked ) return;

  int y = event-{\rangle}xbutton.y;
  int x = event-{\rangle}xbutton.x;
  ulong t = (ulong ) event-{\rangle}xbutton.time;


  int tx = x;
  int ty = y;
  coordinate{\underline}to{\underline}text({\dollar},{\ampersand}tx,{\ampersand}ty);

  /* drag startete vor über einer sekunde, seitdem keine änderung */
  /* darum: vergessen und neu starten */
  if( {\dollar}drag{\underline}update + 1000 {\langle} t ) goto restart{\underline}drag;
  {\dollar}drag{\underline}update = t;

  if( selection{\underline}resize({\dollar},tx,ty) ) {\lbrace}
      {\dollar}drag{\underline}last{\underline}x = x;
      {\dollar}drag{\underline}last{\underline}y = y;
      normalize{\underline}pos({\dollar});
      selection{\underline}draw({\dollar});
  {\rbrace}
  return; /* drag in progress */

  restart{\underline}drag:
  {\dollar}drag{\underline}last{\underline}y = event-{\rangle}xbutton.y;
  {\dollar}drag{\underline}last{\underline}x = event-{\rangle}xbutton.x;
  {\dollar}drag{\underline}update = t;
  selection{\underline}start({\dollar},tx,ty);
  selection{\underline}draw({\dollar});
{\rbrace}\endCode


\End\Imports
\Section
\Code
{\incl} {\langle}stdint.h{\rangle}\endCode


\Section
\Code
{\incl} {\langle}X11/Xft/Xft.h{\rangle}\endCode


\Section
\Code
{\incl} {\langle}X11/Xaw/XawImP.h{\rangle}\endCode


\Section
\Code
{\incl} {\langle}X11/Xatom.h{\rangle}\endCode


\Section
\Code
{\incl} {\langle}X11/Xmu/Xmu.h{\rangle}\endCode


\Section
\Code
{\incl} "xutil.h"\endCode


\Section
\Code
{\incl} "converters.h"\endCode


\Section
\Code
{\incl} "mls.h"\endCode


\End\Privatevars
\Section
\Code
int  drag{\underline}last{\underline}y\endCode


\Section
\Code
int  drag{\underline}last{\underline}x\endCode


\Section
\Code
int  drag{\underline}update\endCode


\Section
\Code
int  top{\underline}x\endCode


\Section
\Code
int  top{\underline}y\endCode


\Section
\Code
int  grid{\underline}pix{\underline}width\endCode


\Section
\Code
int  grid{\underline}pix{\underline}height\endCode


\Section
\Code
int  value\endCode


\Section
\Code
Bool  do{\underline}init\endCode


\Section
\Code
XftDraw * draw\endCode


\Section
\Code
Bool  selection\endCode


\Section
\Code
Bool  selection{\underline}visible\endCode


\Section
\Code
int  selectionText\endCode


\Section
\Code
XRectangle  rsel\endCode


\Section
\Code
XRectangle  rsel{\underline}old\endCode


\Section
\Code
 debug\endCode


\End\Classvars
\Section
\Code
Atom  XA{\underline}UTF8{\underline}STRING = 0 \endCode


\End\Methods
\Section
\Code
initialize(Widget  request, {\dollar}, ArgList  args, Cardinal * num{\underline}args)
{\lbrace}
    if(! {\dollar}XA{\underline}UTF8{\underline}STRING )
        {\dollar}XA{\underline}UTF8{\underline}STRING = XInternAtom(XtDisplay({\dollar}), "UTF8{\underline}STRING", True);

    if( {\dollar}lines == 0 ) {\dollar}lines = m{\underline}create(10,sizeof(char*));
    {\dollar}selectionText = m{\underline}create(100,1);

    {\dollar}do{\underline}init = 1;
    {\dollar}top{\underline}x = {\dollar}top{\underline}y = 0;
    {\dollar}rsel.x = 0;
    {\dollar}rsel.y = 0;
    {\dollar}rsel.width = 1;
    {\dollar}rsel.height = 1;
    {\dollar}rsel{\underline}old = {\dollar}rsel;
    {\dollar}selection = True;
    {\dollar}selection{\underline}visible = True;

    int w = {\dollar}xftFont-{\rangle}ascent + {\dollar}xftFont-{\rangle}descent;
    {\dollar}grid{\underline}pix{\underline}height  = w;
    {\dollar}grid{\underline}pix{\underline}width = xft{\underline}textwidth(XtDisplay({\dollar}), {\dollar}xftFont, "w" );

    if( {\dollar}height == 0 ) {\dollar}height = {\dollar}grid{\underline}pix{\underline}height * {\dollar}min{\underline}lines;
    if( {\dollar}width == 0  ) {\dollar}width = 20;
{\rbrace}\endCode


\Section
marker setzen damit das folgende expose das Window
  neu zeichnen kann, resize wird beim initialisieren der application
  vor realize aufgerufen daher die abfrage ob es sicher ist das
  window zu bearbeiten.


\Code
resize({\dollar})
{\lbrace}
    /* if( XtIsRealized({\dollar}) ) reshape{\underline}widget({\dollar}); */
    {\dollar}do{\underline}init=1;
{\rbrace}\endCode


\Section
\Code
realize({\dollar}, XtValueMask * mask, XSetWindowAttributes * attributes)
{\lbrace}
    Display *dpy = XtDisplay({\dollar});
    XtCreateWindow({\dollar}, (unsigned int) InputOutput,
                          (Visual *) CopyFromParent, *mask, attributes);
    {\dollar}draw = XftDrawCreate(dpy, XtWindow({\dollar}),
                       DefaultVisual(dpy, DefaultScreen(dpy)), None);
{\rbrace}\endCode


\Section
\Code
expose({\dollar}, XEvent * event, Region  region)
{\lbrace}
        if( {\dollar}do{\underline}init ) {\lbrace}
            if( {\dollar}auto{\underline}resize ) {\lbrace}
              {\dollar}grid{\underline}width = {\dollar}width / {\dollar}grid{\underline}pix{\underline}width;
              {\dollar}grid{\underline}height = {\dollar}height / {\dollar}grid{\underline}pix{\underline}height;
            {\rbrace}
            {\dollar}do{\underline}init=0;
         {\rbrace}

        XClearWindow(  XtDisplay({\dollar}), XtWindow({\dollar}) );
        redraw{\underline}full({\dollar});
{\rbrace}\endCode


\End\Utilities
\Section
\Code
clip{\underline}rect({\dollar}, XRectangle * r)
{\lbrace}
    int x0,y0,x1,y1;
    x0 = r-{\rangle}x;
    y0 = r-{\rangle}y;
    x1 = x0 + r-{\rangle}width;
    y1 = y0 + r-{\rangle}height;

    /* oben links ausserhalb */
    if( x0 {\rangle}= {\dollar}grid{\underline}width {\bar}{\bar} y0 {\rangle}= {\dollar}grid{\underline}height ) goto empty{\underline}rect;

    /* unten rechts ausserhalb */
    if( y1 {\langle} 0 {\bar}{\bar} x1 {\langle} 0 ) goto empty{\underline}rect;

    /* auf window zurechtschneiden */
    if( x0 {\langle} 0 ) x0=0;
    if( y0 {\langle} 0 ) y0=0;
    if( x1 {\rangle} {\dollar}grid{\underline}width ) x1 = {\dollar}grid{\underline}width;
    if( y1 {\rangle} {\dollar}grid{\underline}height) y1 = {\dollar}grid{\underline}height;

    r-{\rangle}x = x0; r-{\rangle}y = y0;
    r-{\rangle}width = x1-x0;
    r-{\rangle}height = y1-y0;
    return;

 empty{\underline}rect:
    memset(r, 0, sizeof(*r));
{\rbrace}\endCode


\Section
\Code
draw{\underline}sel{\underline}rect({\dollar})
{\lbrace}
    int x,y,x1,y1;
    XRectangle r1;
    r1 = {\dollar}rsel; /* neu */
    r1.x -= {\dollar}top{\underline}x;
    r1.y -= {\dollar}top{\underline}y;
    x1 = r1.x + r1.width;
    y1 = r1.y + r1.height;

    XRectangle r2;
    r2 = {\dollar}rsel{\underline}old; /* alt */
    r2.x -= {\dollar}top{\underline}x;
    r2.y -= {\dollar}top{\underline}y;

    /* neu */
    for( x=r1.x; x {\langle} x1; x++ )
        for( y=r1.y; y {\langle} y1; y++ ) {\lbrace}
            if( rect{\underline}is{\underline}inside( {\ampersand}r2,x,y ) ) continue;
            draw{\underline}glyph({\dollar},x,y);
        {\rbrace}

    /* alt */
    x1 = r2.x + r2.width;
    y1 = r2.y + r2.height;
    for( x=r2.x; x {\langle} x1; x++ )
        for( y=r2.y; y {\langle} y1; y++ ) {\lbrace}
            if( rect{\underline}is{\underline}inside( {\ampersand}r1,x,y ) ) continue;
            draw{\underline}glyph({\dollar},x,y);
        {\rbrace}
    return;
{\rbrace}\endCode


\Section
\Code
draw{\underline}rect({\dollar}, XRectangle * r)
{\lbrace}
    int x,x1 = r-{\rangle}x + r-{\rangle}width;
    int y,y1 = r-{\rangle}y + r-{\rangle}height;
    for( x=r-{\rangle}x; x {\langle} x1; x++ )
        for( y=r-{\rangle}y; y {\langle} y1; y++ )
            draw{\underline}glyph({\dollar},x,y);
{\rbrace}\endCode


\Section
\Code
selection{\underline}draw({\dollar})
{\lbrace}

    if( {\dollar}selection{\underline}visible ) {\lbrace}
        draw{\underline}sel{\underline}rect({\dollar});
    {\rbrace} else {\lbrace}
        {\dollar}selection{\underline}visible=True;
        XRectangle r = {\dollar}rsel;
        r.x -= {\dollar}top{\underline}x;
        r.y -= {\dollar}top{\underline}y;
        draw{\underline}rect({\dollar},{\ampersand}r);

    {\rbrace}
    {\dollar}rsel{\underline}old = {\dollar}rsel;
{\rbrace}\endCode


\Section
\Code
selection{\underline}clear{\underline}a({\dollar})
{\lbrace}
    if( {\dollar}selection{\underline}visible ) {\lbrace}
        {\dollar}selection{\underline}visible=False;
        memset( {\ampersand} {\dollar}rsel, 0, sizeof( {\dollar}rsel ));
        XRectangle r = {\dollar}rsel{\underline}old;
        r.x -= {\dollar}top{\underline}x;
        r.y -= {\dollar}top{\underline}y;
        draw{\underline}rect({\dollar},{\ampersand}r);
    {\rbrace}
{\rbrace}\endCode


\Section
\Code
redraw{\underline}full({\dollar})
{\lbrace}
        int x; int y;
        for( x=0; x {\langle} {\dollar}grid{\underline}width; x++ ) {\lbrace}
            for(y=0; y {\langle} {\dollar}grid{\underline}height; y++ ) {\lbrace}
                draw{\underline}glyph({\dollar},x,y );
            {\rbrace}
        {\rbrace}
{\rbrace}\endCode


\Section
\Code
uint32{\underline}t  find{\underline}charpos({\dollar}, int  x, int  y, int * ret{\underline}line, int * ret{\underline}charpos)
{\lbrace}
    int ln = {\dollar}top{\underline}y + y;
    int len, p  = {\dollar}top{\underline}x + x;
    int l;
    uint8{\underline}t *string;
    uint32{\underline}t ucs4;

    *ret{\underline}line = -1;
    *ret{\underline}charpos = -1;

    if( ln {\rangle}= m{\underline}len({\dollar}lines) ) return 0;
    if(ret{\underline}line) *ret{\underline}line = ln;
    string = (uint8{\underline}t *)STR({\dollar}lines,ln);
    len = strlen( (char*)string);
    while(1) {\lbrace}
            l = FcUtf8ToUcs4 (string, {\ampersand}ucs4, len);
            if( l{\langle}= 0 {\bar}{\bar} len {\langle}=0 ) return 0;
            if( p {\langle}=0 ) {\lbrace}
                if( ret{\underline}charpos ) *ret{\underline}charpos = string - (uint8{\underline}t*)STR({\dollar}lines,ln);
                return ucs4;
            {\rbrace}

            string += l;
            len -= l;
            p--;
    {\rbrace}

{\rbrace}\endCode


\Section
\Code
uint32{\underline}t  get{\underline}ucs4({\dollar}, int  x, int  y)
{\lbrace}

    int d1,d2;
    return find{\underline}charpos({\dollar},x,y,{\ampersand}d1,{\ampersand}d2);
{\rbrace}\endCode


\Section
\Code
draw{\underline}glyph({\dollar}, int  x, int  y)
{\lbrace}
    if( x {\langle} 0 {\bar}{\bar} x {\rangle}= {\dollar}grid{\underline}width ) return;
    if( y {\langle} 0 {\bar}{\bar} y {\rangle}= {\dollar}grid{\underline}height ) return;

    int fg = 0;
    int bg = 3;
    if( {\dollar}selection {\ampersand}{\ampersand} rect{\underline}is{\underline}inside( {\ampersand} {\dollar}rsel, x + {\dollar}top{\underline}x, y + {\dollar}top{\underline}y ) ) {\lbrace}
        fg = 1;
        bg = 4;
    {\rbrace}

    draw{\underline}glyph{\underline}color({\dollar},x,y,fg,bg);
{\rbrace}\endCode


\Section
\Code
draw{\underline}glyph{\underline}color({\dollar}, int  x, int  y, int  fg, int  bg)
{\lbrace}
    int x0,y0,w,h;
    uint32{\underline}t ucs4, glyph;

    x0 = x * {\dollar}grid{\underline}pix{\underline}width;
    y0 = y * {\dollar}grid{\underline}pix{\underline}height;
    w = {\dollar}grid{\underline}pix{\underline}width;
    h = {\dollar}grid{\underline}pix{\underline}height;

    XftDrawRect({\dollar}draw,{\dollar}xft{\underline}col+bg, x0,y0,w,h );
    ucs4 = get{\underline}ucs4({\dollar}, x, y );
    if( ucs4 == 0 {\bar}{\bar} ucs4 == 32 ) return;
    glyph = XftCharIndex ( XtDisplay({\dollar}), {\dollar}xftFont, ucs4 );
    XftDrawGlyphs ({\dollar}draw, {\dollar}xft{\underline}col+fg, {\dollar}xftFont,
                   x0, y0+{\dollar}xftFont-{\rangle}ascent,
                   {\ampersand}glyph, 1);
{\rbrace}\endCode


\Section
\Code
String  get{\underline}name({\dollar})
{\lbrace}
          return "";
{\rbrace}\endCode


\Section
\Code
void  normalize{\underline}pos({\dollar})
{\lbrace}
        if( {\dollar}top{\underline}x {\langle} 0 ) {\dollar}top{\underline}x = 0;
        if( {\dollar}top{\underline}y {\langle} 0 ) {\dollar}top{\underline}y = 0;
{\rbrace}\endCode


\Section
\Code
Bool  coordinate{\underline}to{\underline}text({\dollar}, int * x, int * y)
{\lbrace}
    Bool ret = (*x {\rangle}= 0) {\ampersand}{\ampersand} (*x {\langle} {\dollar}width) {\ampersand}{\ampersand} (*y {\rangle} 0) {\ampersand}{\ampersand} (*y {\langle} {\dollar}height);

    *x /= {\dollar}grid{\underline}pix{\underline}width;
    *x += {\dollar}top{\underline}x;
    *y /= {\dollar}grid{\underline}pix{\underline}height;
    *y += {\dollar}top{\underline}y;

    return ret;

{\rbrace}\endCode


\Section
\Code
selection{\underline}start({\dollar}, int  x, int  y)
{\lbrace}
    {\dollar}selection=True;
    {\dollar}rsel.x = x; {\dollar}rsel.width=1;
    {\dollar}rsel.y = y; {\dollar}rsel.height=1;
{\rbrace}\endCode


\Section
\Code
Bool  selection{\underline}resize({\dollar}, int  x, int  y)
{\lbrace}
    Bool changed = False;
    int w = {\dollar}rsel.width;
    int h = {\dollar}rsel.height;
    int x0 = {\dollar}rsel.x;
    int y0 = {\dollar}rsel.y;
    int x1 = x0 + w -1;
    int y1 = y0 + h -1;

    if( x {\rangle} x0 ) x1 = x; else {\lbrace} x1 = x0; x0 = x; {\rbrace}
    if( y {\rangle} y0 ) y1 = y; else {\lbrace} y1 = y0; y0 = y; {\rbrace}

    w = x1 - x0 +1;
    h = y1 - y0 +1;
    if( {\dollar}rsel.width != w ) {\lbrace}
        changed = True;
        {\dollar}rsel.width = w;
    {\rbrace}
    if( {\dollar}rsel.height != h ) {\lbrace}
        changed = True;
        {\dollar}rsel.height = h;
    {\rbrace}
    if( {\dollar}rsel.x != x0 ) {\lbrace}
        changed = True;
        {\dollar}rsel.x = x0;
    {\rbrace}
    if( {\dollar}rsel.y != y0 ) {\lbrace}
        changed = True;
        {\dollar}rsel.y = y0;
    {\rbrace}


    return changed;
{\rbrace}\endCode


\Section
\Code
draw{\underline}line{\underline}from({\dollar}, int  x, int  y)
{\lbrace}

    x -= {\dollar}top{\underline}x;
    y -= {\dollar}top{\underline}y;
    while( x {\langle} {\dollar}grid{\underline}width ) {\lbrace}
        draw{\underline}glyph({\dollar},x,y);
        x++;
    {\rbrace}
{\rbrace}\endCode


\Section
\Code
int  utf8{\underline}strlen(char * s)
{\lbrace}
    int n,w;
    FcUtf8Len((uint8{\underline}t*)s,strlen(s), {\ampersand}n, {\ampersand}w );
    return n;
{\rbrace}\endCode


\Section
\Code
remove{\underline}char{\underline}at{\underline}cursor({\dollar})
{\lbrace}
    uint32{\underline}t ucs4;
    int x,y,p,byte{\underline}len,byte{\underline}pos,w;
    x={\dollar}rsel.x;
    y={\dollar}rsel.y;
    if( y {\rangle}= m{\underline}len({\dollar}lines) ) return;
    char *s = STR({\dollar}lines,y);
    if( x {\rangle}= utf8{\underline}strlen(s) ) return;

    p=0;
    byte{\underline}len = strlen(s) +1;
    byte{\underline}pos = 0;
    while(1) {\lbrace}
        w = FcUtf8ToUcs4 ((uint8{\underline}t*)s + byte{\underline}pos, {\ampersand}ucs4, byte{\underline}len - byte{\underline}pos);
        if( w {\langle}=0 ) return;
        if( p {\rangle}= x ) break;
        byte{\underline}pos+=w; p++;
    {\rbrace}

    memcpy( s+byte{\underline}pos, s+byte{\underline}pos + w, byte{\underline}len - byte{\underline}pos - w );
    STR({\dollar}lines,y) = realloc(s, byte{\underline}len - w);
{\rbrace}\endCode


\Section
\Code
insert{\underline}char{\underline}at{\underline}cursor({\dollar}, char * buf, int  len)
{\lbrace}
    uint32{\underline}t ucs4;
    int p;
    int x,y,w;
    char *s;
    int byte{\underline}len;
    int utf8{\underline}len;
    int byte{\underline}pos;

    x={\dollar}rsel.x;
    y={\dollar}rsel.y;

    while( y {\rangle}= m{\underline}len({\dollar}lines) ) {\lbrace}
        s=strdup("");
        m{\underline}put( {\dollar}lines, {\ampersand}s );
    {\rbrace}


    s=STR({\dollar}lines,y);
    byte{\underline}len = strlen(s);
    utf8{\underline}len = utf8{\underline}strlen(s);

    /* append data */
    if( x {\rangle}= utf8{\underline}len ) {\lbrace}
        /* fill byte{\underline}len .. x */
        int n = x - utf8{\underline}len; /* append space */
        s = realloc(s,byte{\underline}len+n+len+1);
        while( n-- ) s[byte{\underline}len++] = 32;
        memcpy( s+byte{\underline}len, buf, len );
        s[byte{\underline}len+len] = 0;
        STR({\dollar}lines,y) = s;
        return;
    {\rbrace}

    /* insert data */
    p=0;
    byte{\underline}len = strlen(s) +1;
    byte{\underline}pos = 0;
    while( p {\langle} x ) {\lbrace}
        w = FcUtf8ToUcs4 ((uint8{\underline}t*)s + byte{\underline}pos, {\ampersand}ucs4, byte{\underline}len - byte{\underline}pos);
        if( w {\langle}=0 ) break;
        byte{\underline}pos+=w; p++;
    {\rbrace}
    s = realloc(s, byte{\underline}len+len);
    STR({\dollar}lines,y) = s;
    memcpy( s + byte{\underline}pos + len, s+byte{\underline}pos, byte{\underline}len - byte{\underline}pos );
    memcpy( s + byte{\underline}pos, buf, len );
{\rbrace}\endCode


\Section
\Code
cursor{\underline}right({\dollar})
{\lbrace}
    {\dollar}rsel.x++;
    if( {\dollar}rsel.x {\rangle}= {\dollar}grid{\underline}width ) {\lbrace}
        {\dollar}top{\underline}x = {\dollar}rsel.x - {\dollar}grid{\underline}width + 1;
        redraw{\underline}full({\dollar});
    {\rbrace}
    selection{\underline}draw({\dollar});
{\rbrace}\endCode


\Section
\Code
Bool  cursor{\underline}pos1({\dollar})
{\lbrace}
    Bool redraw;
    redraw = ( {\dollar}top{\underline}x != 0 );
    {\dollar}top{\underline}x = 0;
    {\dollar}rsel.x=0;
    return redraw;
{\rbrace}\endCode


\Section
\Code
Bool  cursor{\underline}down({\dollar})
{\lbrace}
    Bool redraw;
    {\dollar}rsel.y++;
    if( (redraw={\dollar}rsel.y {\rangle}= {\dollar}grid{\underline}height) ) {\lbrace}
        {\dollar}top{\underline}y = {\dollar}rsel.y - {\dollar}grid{\underline}height + 1;
    {\rbrace}
    return redraw;
{\rbrace}\endCode


\Section
\Code
Bool  current{\underline}line{\underline}empty({\dollar})
{\lbrace}
    int ln = {\dollar}rsel.y;
    if( ln {\rangle}= m{\underline}len({\dollar}lines) ) return True;
    char *s = STR({\dollar}lines,ln);
    return *s == 0;
{\rbrace}\endCode


\Section
\Code
Bool  remove{\underline}current{\underline}line({\dollar})
{\lbrace}
    int ln = {\dollar}rsel.y;
    if( ln {\rangle}= m{\underline}len({\dollar}lines) ) return False;
    free( STR({\dollar}lines, ln) );
    m{\underline}del({\dollar}lines,ln);
    return True;
{\rbrace}\endCode


\Section
\Code
Bool  insert{\underline}line({\dollar})
{\lbrace}
    char *s;
    int y = {\dollar}rsel.y;
    if( y {\langle} m{\underline}len({\dollar}lines) ) {\lbrace}
        m{\underline}ins({\dollar}lines,y,1);
        STR({\dollar}lines,y) = strdup("");
    {\rbrace} else {\lbrace}
        while( y {\rangle}= m{\underline}len({\dollar}lines) ) {\lbrace}
            s=strdup("");
            m{\underline}put( {\dollar}lines, {\ampersand}s );
        {\rbrace}
    {\rbrace}
    return True;

{\rbrace}\endCode


\Section
\Code
char * utf8{\underline}skip(char * str, int  cnt)
{\lbrace}
    uint8{\underline}t *s = (uint8{\underline}t *) str;
    while( cnt {\ampersand}{\ampersand} *s ) {\lbrace}
        cnt--;
        s++;
        while( (*s {\ampersand} 0b11000000) == 0b10000000 ) s++; /* utf8 */
    {\rbrace}
    return (char*) s;
{\rbrace}\endCode


\Section
\Code
Bool  split{\underline}line{\underline}at{\underline}cursor({\dollar})
{\lbrace}
    int ln = {\dollar}rsel.y;

    if( ln {\rangle}= m{\underline}len({\dollar}lines) ) return False;     /* kein text */

    char *s = STR({\dollar}lines,ln);
    int utf8{\underline}len = utf8{\underline}strlen(s);
    char *s2, *new{\underline}line;
    int crsr{\underline}x = {\dollar}rsel.x;

    if( {\dollar}rsel.x {\langle} utf8{\underline}len ) {\lbrace}     /* innerhalb einer zeile */
        s2 = utf8{\underline}skip(s, crsr{\underline}x);
        new{\underline}line = strdup(s2);
        *s2 = 0;
        s = realloc(s, s2-s +1 );
        STR({\dollar}lines,ln) = s;
    {\rbrace} else new{\underline}line = strdup("");  /* am ende einer zeile */

    ln++;
    m{\underline}ins({\dollar}lines,ln,1);
    STR({\dollar}lines,ln) = new{\underline}line;
    return True;
{\rbrace}\endCode


\Section
callback function, called by XtGetSelectionValue()


\Code
RequestSelection({\dollar}, XtPointer  client, Atom * selection, Atom * type, XtPointer  value, unsigned  long * length, int * format)
{\lbrace}
    char *s = value;
    int len = *length;
    if( is{\underline}empty(s) {\bar}{\bar} len {\langle}= 0 ) return;
    insert{\underline}char{\underline}at{\underline}cursor({\dollar},s,len);
    while( utf8char({\ampersand}s) != -1 ) cursor{\underline}right({\dollar});
    draw{\underline}line{\underline}from({\dollar}, {\dollar}rsel.x, {\dollar}rsel.y);
{\rbrace}\endCode


\Section
\Code
utf8{\underline}copy{\underline}char(int  buf, char ** strp)
{\lbrace}
    char *s = *strp;
    do {\lbrace}
        m{\underline}putc(buf,*s++);
    {\rbrace}
    while( (*s {\ampersand} 0b11000000) == 0b10000000 ); /* utf8 */
    *strp = s;
{\rbrace}\endCode


\Section
\Code
Boolean  ConvertSelection({\dollar}, Atom * selection, Atom * target, Atom * type, XtPointer * value, unsigned  long * length, int * format)
{\lbrace}

  XSelectionRequestEvent *req = XtGetSelectionRequest({\dollar}, *selection, NULL);

  /* client ask for possible types */
  if (*target == XA{\underline}TARGETS(XtDisplay({\dollar}))) {\lbrace}
      Atom *targetP, *std{\underline}targets;
      unsigned long std{\underline}length;

      /* get possible targets */
      XmuConvertStandardSelection({\dollar}, req-{\rangle}time, selection,
                                  target, type, (XPointer *) {\ampersand} std{\underline}targets,
                                  {\ampersand}std{\underline}length, format);

      *value = XtMalloc((unsigned) sizeof(Atom) * (std{\underline}length + 1));
      targetP = *(Atom **) value;

      *length = std{\underline}length + 1;     /* list of standard types */
      *targetP++ = {\dollar}XA{\underline}UTF8{\underline}STRING;  /* prefered target */
      /* append standard targets */
      memmove((char *) targetP, (char *) std{\underline}targets, sizeof(Atom) * std{\underline}length);
      XtFree((char *) std{\underline}targets); /* remove original list */
      *type = XA{\underline}ATOM;
      *format = sizeof(Atom) * 8;
      return True;
  {\rbrace}
  else if (*target == {\dollar}XA{\underline}UTF8{\underline}STRING) {\lbrace}
      *length = (long) m{\underline}len({\dollar}selectionText);
      if( *length ) *value = strdup(m{\underline}buf({\dollar}selectionText));
      else *value = 0;
      *type = {\dollar}XA{\underline}UTF8{\underline}STRING;
      *format = 8;
      return True;
  {\rbrace}
  return False;
{\rbrace}\endCode


\Section
\Code
LoseSelection({\dollar}, Atom * selection)
{\lbrace}
  {\dollar}rsel.width=1;
  {\dollar}rsel.height=1;
  selection{\underline}draw({\dollar});
{\rbrace}\endCode


\Section
\Code
own{\underline}primary{\underline}selection({\dollar}, ulong  time)
{\lbrace}

    int ln = {\dollar}rsel.y;
    if( ln {\rangle}= m{\underline}len({\dollar}lines) ) return;     /* kein text */
    char *s = STR({\dollar}lines,ln);
    int utf8{\underline}len = utf8{\underline}strlen(s);
    if( {\dollar}rsel.x {\rangle}= utf8{\underline}len ) return;     /* kein text */
    int char{\underline}cnt = {\dollar}rsel.width;

    /* clear buffer */
    m{\underline}clear( {\dollar}selectionText );
    /* ptr to start of selection */
    s = utf8{\underline}skip(s, {\dollar}rsel.x );
    /* copy selection */
    while( char{\underline}cnt ) {\lbrace}
        utf8{\underline}copy{\underline}char({\dollar}selectionText, {\ampersand}s);
        char{\underline}cnt--;
    {\rbrace}
    m{\underline}putc({\dollar}selectionText,0);

    XtOwnSelection({\dollar}, XA{\underline}PRIMARY, time,
                   ConvertSelection, LoseSelection, NULL);
    XChangeProperty(XtDisplay({\dollar}), DefaultRootWindow(XtDisplay({\dollar})),
                    XA{\underline}CUT{\underline}BUFFER0, {\dollar}XA{\underline}UTF8{\underline}STRING, 8, PropModeReplace,
                    (unsigned char *) m{\underline}buf({\dollar}selectionText),
                    m{\underline}len({\dollar}selectionText));
{\rbrace}\endCode


\End\bye
